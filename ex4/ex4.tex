\documentclass{article}

\usepackage{etex}
\usepackage[utf8]{inputenc}
\usepackage{csquotes}
\usepackage{ngerman}
\usepackage{graphicx}
\usepackage{amssymb}
\usepackage{fancyhdr}
\usepackage{caption}
\usepackage{url}
\usepackage{lastpage}
\usepackage{geometry}
\usepackage{amsmath}
\usepackage{titlesec}
\usepackage{amsthm}
\usepackage{mathtools}
\usepackage{extarrows}
\usepackage{listings} \lstset{numbers=left, numberstyle=\tiny, numbersep=5pt} \lstset{language=Haskell}
\usepackage{stmaryrd}
%Felix: There is a conflict between mathtools and xfraf whem I'm compiling the
% document.
% https://lists.debian.org/debian-tex-maint/2011/03/msg00101.html
% \usepackage{xfrac}
\usepackage{tikz}
\usepackage{epstopdf}
\usepackage{float}
\usepackage{stmaryrd}
\usepackage{centernot}
\usepackage{amssymb}
\usepackage{calc}
\usepackage[nomessages]{fp}
\usepackage{polynom}
\usepackage{ctable}
\usepackage[sharp]{easylist}
\usepackage{siunitx}
\usepackage{pdfpages}
\usepackage{enumerate}
\usepackage{algorithm}
\usepackage{algpseudocode}
\usepackage{pifont}
\usepackage{hyperref}
\usepackage{placeins}
\usepackage[makeroom]{cancel}
%\usepackage{sagetex}

\geometry{a4paper,left=3cm, right=3cm, top=3cm, bottom=3cm}
\pagestyle{fancy}
\fancyhead[C]{Codierungstheorie}
\fancyhead[R]{\today}
\fancyfoot[L]{Simon Wegendt}
\fancyfoot[C]{Seite \thepage /\pageref{LastPage}}
\fancyfoot[R]{David Binder}
\renewcommand{\headrulewidth}{0.4pt}
\renewcommand{\footrulewidth}{0.4pt}
\newcommand{\cmark}{\text{\ding{51}}}%
\newcommand{\xmark}{\text{\ding{55}}}%
\parindent0pt
%\newcommand{\N}{\mbox{$I\!\!N$}}
%\newcommand{\Z}{\mbox{$Z\!\!\!Z$}}
%\newcommand{\Q}{\mbox{$I\:\!\!\!\!\!Q$}}
%\newcommand{\R}{\mbox{$I\!\!R$}}
\newcommand{\N}{\mathbb{N}}
\newcommand{\Z}{\mathbb{Z}}
\newcommand{\Q}{\mathbb{Q}}
\newcommand{\R}{\mathbb{R}}
\newcommand{\C}{\mathcal{C}}
\newcommand{\D}{\mathbb{D}}
\renewcommand{\L}{\mathbb{L}}
\renewcommand{\P}{\mathcal{P}}
\newcommand{\set}[1]{\left\lbrace #1 \right\rbrace}
\newcommand{\tuple}[1]{\left( #1 \right)}
\newcommand{\entspr}{\ensuremath \widehat{=}}
\newcommand{\innervect}[1]{\begin{array}{c}#1\end{array}}
\newcommand{\vect}[1]{\tuple{\innervect{#1}}}
\makeatletter
\newcommand*{\rom}[1]{\expandafter\@slowromancap\romannumeral #1@}
\makeatother
\newcommand{\lgsto}[4]{
  \xrightarrow[
    \overset{
      \text{\scriptsize #3}
    }{
      \text{#4}
    }
  ]{
    \overset{
      \text{\scriptsize #1}
    }{
      \text{#2}
    }
  }
}
\renewcommand{\mod}{\text{ mod }}
\newcommand{\ggT}{\text{ggT}}
\newcommand{\id}{\text{id}}
\newcommand{\IV}{\overset{\text{IV}}{=}}
\newcommand{\ZZ}{\ensuremath{\mathrm{Z\kern-.3em\raise-0.5ex\hbox{Z}}:\phantom{}}}
\newcommand{\grad}{\text{grad}}
\newcommand{\phantomn}{\phantom{}}
\newcommand{\overtext}[2]{\overset{\text{\scriptsize #1}}{#2}}
\newcommand{\Abb}{\text{Abb}}
\newcommand{\seilpmi}{\hspace{3pt}\Longleftarrow\hspace{3pt}}
\newcommand{\Kern}{\text{Kern}}
\newcommand{\Bild}{\text{Bild}}
\newcommand{\Spann}[1]{\left\langle#1\right\rangle}
\newcommand{\Rang}{\text{Rang}}
\renewcommand{\phi}{\varphi}
\newcommand{\ceil}[1]{\left\lceil#1\right\rceil}
\newcommand{\floor}[1]{\left\lfloor#1\right\rfloor}
\newcommand{\TM}[1]{\left\langle#1\right\rangle}
\newcommand{\ot}{\reflectbox{$\to$}}
\newcommand{\underscore}{\underline{\hspace{5pt}}}
\newcommand{\VR}{\text{VR}}

\DeclareMathOperator*{\argmax}{argmax\,}
\DeclareMathOperator*{\argmin}{argmin\,}

\def\splitstring#1#2{%
    \noexpandarg
    \IfSubStr{#1}{#2}{
    \StrBefore{#1}{#2}
    }{#1}
}
\newcommand{\firstline}[1]{\splitstring{#1}{\\}}

\newcommand{\listSettings}{\ListProperties(Numbers1=l, Mark1={)}, Style1*=\bfseries\large, Numbers2=a, Numbers3=R, Numbers4=r, Hide3=2, Mark2=., Style2*={}, Progressive=1em)}

\newcommand{\autobox}[1]{\parbox{\widthof{\firstline{#1}}}{#1}}
\newenvironment{myList}{
  \begin{easylist}[enumerate]\listSettings
}{\end{easylist}}

\newcommand*{\independent}{\ensuremath{\bot\hspace{-0.5em}\bot}}
\newcommand*{\given}{\,|\,}


\fancyhead[L]{Übungsblatt 4}
\setcounter{MaxMatrixCols}{20}
\begin{document}
\section*{Aufgabe 17}
Sei $\C$ ein linearer $[n,k]$-Code über dem endlichen Körper $K$ mit Kontrollmatrix $H$.
Sei $y \in K^n \setminus \C$. Zeigen Sie:
\begin{myList}
#
Genau dann existiert ein $c \in \C$ mit $d(c,y)=1$, wenn $Hy^t = a \cdot s$ für ein $a \in K \setminus \lbrace 0 \rbrace$ und eine Spalte $s$ von $H$.\\
Ist dabei $s$ die $i$-te Spalte von $H$, so gilt $d(c,y) = 1$ für $c = y -a\cdot e_i \in \C$, wobei $e_i$ der $i$-te kanonische Basisvektor des $K^n$ ist.
#
Die Anzahl der Spalten von $H$, die ein von 0 verschiedenes Vielfaches von $Hy^t$ sind, ist gleich der Anzahl der Codewörter $c$ mit $d(c,y) = 1$.
#
Entwickeln Sie unter Verwendung von a) ein schnelles Decodierverfahren für die Hamming-Codes.
\end{myList}
\section*{Aufgabe 18}
Sei $\C = \lbrace (x_1,x_2,x_3,x_4,x_5) \in \Z^5_7 : x_4 = 2x_1 + 4x_3, x_5 = x_1 + 2x_2 + 2x_3 \rbrace$
Bestimmen Sie Erzeuger und Kontrollmatrizen für $\C$ und $\C^\bot$.
\section*{Aufgabe 19}
Sei $\C$ der $[7,4]$-Code über $\Z_3$ mit Erzeugermatrix
\begin{equation*}
	G =
	\begin{pmatrix}
	2 & 1 & 0 & 1 & 1 & 0 & 0 \\
	2 & 1 & 0 & 2 & 0 & 2 & 0 \\
	0 & 2 & 0 & 2 & 0 & 2 & 1 \\
	0 & 1 & 1 & 1 & 0 & 2 & 0
	\end{pmatrix}
\end{equation*}
\begin{myList}
#
Zeigen Sie, dass $\C$ eine Erzeugermatrix in Standardform besitzt.
#
Bestimmen Sie eine Erzeugermatrix von $C^\bot$.
#
Bestimmen Sie $C \cap C^\bot$.
\end{myList}
\section*{Aufgabe 20}
\begin{myList}
#
Bestimmen Sie alle selbstdualen Hamming-Codes.
(Hinweis: Nicht nur auf den binären Fall beschränken!)\medskip

Wir wollen diejenigen $\C$ bestimmen für die gilt:
\begin{equation*}
	\C = \C^\bot
\end{equation*}
unter der Voraussetzung dass $\C$ Hamming-Code ist.\medskip

Aus Lemma X folgt dass falls $\C$ ein $[n,k]$ Code ist, dann $\C^\bot$ ein $[n,n-k]$ Code ist. Da wir zeigen wollen dass $\C = \C^\bot$ muss also gelten dass $n$ gerade ist und $k = \frac{n}{2}$.
Im folgenden seien also sowohl $\C$ als auch $\C^\bot$ $[2n, n]$ Codes.\medskip

Aus der Tatsache dass $\C$ Hamming Code ist wissen wir dass:
\begin{itemize}
	\item $d(\C) = 3$
	\item $\C$ ist Untervektorraum von $K^{2n}$, $dim(C) = n = 2n - l$ $\Rightarrow l = n$
	\item $\C \subset K^{2n}$, $|K| = q$, $2n = \frac{q^n - 1}{q -1}$ ($q$ ist Primzahlpotenz)
	\item $\C$ ist perfekt, i.e. $|\C| = \frac{q^{2n}}{\sum_{i=0}^1 \binom{2n}{i}(q -1)^i} =  \frac{q^{2n}}{ 1 + 2n(q -1)}$
\end{itemize}\medskip

In der Gleichung $|\C| = \frac{q^{2n}}{1 + 2n(q-1)}$ können wir $2n$ im Nenner durch $\frac{q^n -1}{q-1}$ ersetzen und erhalten $|\C| = q^n$.\medskip

Sowohl die Erzeugermatrix als auch die Kontrollmatrix von $\C$ muss eine $n \times 2n$-Matrix sein.
Desweiteren muss die Erzeugermatrix von $\C$ immer auch eine Kontrollmatrix von $\C$ sein.
Das folgt aus Lemma X welches behauptet dass die Erzeugermatrix von $\C$ eine Kontrollmatrix von $\C^\bot = \C$ ist. 
Aus dem selben Grund muss Ausserdem jede Kontrollmatrix von $\C$ eine Erzeugermatrix von $\C^\bot = \C$ sein.\medskip

Wir haben bisher also gezeigt dass falls $\C$ ein Hamming-Code ist und falls $\C = \C^\bot$, dann muss $\C$ ein $[2n,n]$ Code sein, und die Erzeugermatrix von $\C$ muss auch eine Kontrollmatrix von $\C$ sein. Ausserdem muss $|\C| = q^n$ sein.\medskip

Aus $|\C| = q^n$ und der Tatsache dass die Erzeugermatrix von $\C$ eine $n \times 2n$ Matrix ist folgt aus Kardinalitätsgründen auch dass alle Zeilen der Erzeuger/Kontrollmatrix linear unabhängig sein müssen.

#
Konstruieren Sie für jede gerade natürliche Zahl $n$ einen selbstdualen binären linearen Code der Länge $n$.\medskip

Behauptung: für jede natürliche Zahl $n$ erzeugt die folgende Erzeugermatrix einen selbstdualen binären linearen Code der Länge n:
\begin{equation*}
	G = (I_{n/2} | I_{n/2})
\end{equation*}
Am Beispiel von $n = 12$ sieht $G$ also folgendermaßen aus:
\begin{equation*}
	G = 
	\begin{pmatrix}
	1 & 0 & 0 & 0 & 0 & 0 		& 1 & 0 & 0 & 0 & 0 & 0\\
	0 & 1 & 0 & 0 & 0 & 0 		& 0 & 1 & 0 & 0 & 0 & 0\\
	0 & 0 & 1 & 0 & 0 & 0 		& 0 & 0 & 1 & 0 & 0 & 0\\
	0 & 0 & 0 & 1 & 0 & 0 		& 0 & 0 & 0 & 1 & 0 & 0\\
	0 & 0 & 0 & 0 & 1 & 0 		& 0 & 0 & 0 & 0 & 1 & 0\\
	0 & 0 & 0 & 0 & 0 & 1 		& 0 & 0 & 0 & 0 & 0 & 1
	\end{pmatrix}
\end{equation*}
Wird zB das Infowort $i =(1,0,0,1,0,1)$ codiert so erhält man das Codewort $c = i \cdot G = (1,0,0,1,0,1,1,0,0,1,0,1)$. Und da $G$ auch Kontrollmatrix für $\C$ ist gilt $G \cdot (1,0,0,1,0,1,1,0,0,1,0,1)^t = 0$.\medskip

Allgemein ist zu zeigen dass: $c \in \C \Leftrightarrow c \in \C^\bot$.
Das gilt nach Lemma X, d) da G in Standardform ist und $G = (I_k|A) = (-A^t|I_k)$ für das angegebene $G$ trivialerweise erfüllt ist, $G$ also auch eine Erzeugermatrix des dualen Codes ist.
\end{myList}
\section*{Aufgabe 21}
Sei $\C$ der binäre $[15,4]$-Simplex Code mit Erzeugermatrix
\begin{equation*}
	G = 
	\begin{pmatrix}
	1 & 0 & 0 & 1 & 1 & 0 & 0 & 1 & 1 & 1 & 1 & 1 & 0 & 0 & 0 \\
	1 & 1 & 0 & 0 & 0 & 1 & 1 & 0 & 1 & 1 & 1 & 0 & 1 & 0 & 0 \\
	0 & 1 & 1 & 0 & 1 & 0 & 1 & 1 & 0 & 1 & 1 & 0 & 0 & 1 & 0 \\
	0 & 0 & 1 & 1 & 0 & 1 & 1 & 1 & 1 & 0 & 1 & 0 & 0 & 0 & 1
	\end{pmatrix}
\end{equation*}
\begin{myList}
#
Zeigen Sie, dass $\C$ 3-Fehler-korrigierend ist.

Da $\C$ Simplex-Code ist gilt dass für alle Codewörter $c\neq c'$ in $\C$ der Abstand konstant ist. Insbesondere gilt daher $d(\C) = wt(c)$ für ein beliebiges Codewort $c \neq 0 \in \C$.\medskip

Aus der Erzeugermatrix kann daher direkt abgelesen werden dass $d(\C) = 8$.
Daraus folgt direkt dass das maximale $t$ mit $2t+1 \leq d(\C) = 8$ genau 3 ist.
#
Bei der Übertragung eines Codeworts aus $\C$ seien maximal 3 Fehler aufgetreten. Das empfangene Wort sei:
\begin{equation*}
	z = (0,1,1,1,1,0,0,0,0,0,1,1,1,1,1)
\end{equation*}
Entscheiden Sie mittels (einstufiger) Majority-Logic-Decodierung, ob an Position 4 ein Fehler aufgetreten ist.\medskip

Da maximal 3 Fehler aufgetreten sind müssen wir 6 Vektoren aus $\C^\bot$ finden die orthogonal bezüglich der 4ten Position sind.\medskip

Es gilt in Variation des Beweises von Lemma X:\\
$(A |I_k)$ Erzeugermatrix von $\C \Rightarrow$ $(I_{n-k}|-A^t)$ Erzeugermatrix von $\C^\bot$.
Beweis:
\begin{equation*}
	(I_{n-k}|-A^t) (A|I_k)^t = (I_{n-k}|-A^t)\left(\frac{A^t}{I_k}\right) = I_{n+k}A^t - A^tI_k = 0
\end{equation*}\medskip

Also ist die folgende Matrix Erzeugermatrix von $C^\bot$:
\begin{equation*}
	\begin{pmatrix}
	1 & 0 & 0 & 0 & 0 & 0 & 0 & 0 & 0 & 0 & 0 &		1 & 1 & 0 & 0 \\
	0 & 1 & 0 & 0 & 0 & 0 & 0 & 0 & 0 & 0 & 0 &		0 & 1 & 1 & 0 \\
	0 & 0 & 1 & 0 & 0 & 0 & 0 & 0 & 0 & 0 & 0 &		0 & 0 & 1 & 1 \\
	0 & 0 & 0 & 1 & 0 & 0 & 0 & 0 & 0 & 0 & 0 &		1 & 0 & 0 & 1 \\
	0 & 0 & 0 & 0 & 1 & 0 & 0 & 0 & 0 & 0 & 0 &		1 & 0 & 1 & 0 \\
	0 & 0 & 0 & 0 & 0 & 1 & 0 & 0 & 0 & 0 & 0 &		0 & 1 & 0 & 1 \\
	0 & 0 & 0 & 0 & 0 & 0 & 1 & 0 & 0 & 0 & 0 &		0 & 1 & 1 & 1 \\
	0 & 0 & 0 & 0 & 0 & 0 & 0 & 1 & 0 & 0 & 0 &		1 & 0 & 1 & 1 \\
	0 & 0 & 0 & 0 & 0 & 0 & 0 & 0 & 1 & 0 & 0 &		1 & 1 & 0 & 1 \\
	0 & 0 & 0 & 0 & 0 & 0 & 0 & 0 & 0 & 1 & 0 &		1 & 1 & 1 & 0 \\
	0 & 0 & 0 & 0 & 0 & 0 & 0 & 0 & 0 & 0 & 1 &		1 & 1 & 1 & 1 
	\end{pmatrix}
\end{equation*}
Die Zeilen seien als $z_0$ bis $z_{10}$ bezeichnet, wir geben jetzt $v_0$ bis $v_5$ aus $\C^\bot$ an die orthogonal bzgl. der vierten Position sind:\\
\begin{align*}
	v_0 &= z_3 &= \begin{pmatrix}
		0 & 0 & 0 & 1 & 0 & 0 & 0 & 0 & 0 & 0 & 0 &		1 & 0 & 0 & 1
		\end{pmatrix} \\
	v_1 &= z_3 + z_7 &= \begin{pmatrix}
		0 & 0 & 0 & 1 & 0 & 0 & 0 & 1 & 0 & 0 & 0 &		0 & 0 & 1 & 0
		\end{pmatrix}\\
	v_2 &= z_3 + z_8 &= \begin{pmatrix}
		0 & 0 & 0 & 1 & 0 & 0 & 0 & 0 & 1 & 0 & 0 &		0 & 1 & 0 & 0
		\end{pmatrix}\\
	v_3 &= z_3 + z_1 + z_{10} &= \begin{pmatrix}
		0 & 1 & 0 & 1 & 0 & 0 & 0 & 0 & 0 & 0 & 1 &		0 & 0 & 0 & 0
		\end{pmatrix}\\
	v_4 &= z_3 + z_2 + z_4 &= \begin{pmatrix}
		0 & 0 & 1 & 1 & 1 & 0 & 0 & 0 & 0 & 0 & 0 &		0 & 0 & 0 & 0
		\end{pmatrix}\\
	v_5 &= z_3 + z_0 + z_5 &= \begin{pmatrix}
		1 & 0 & 0 & 1 & 0 & 1 & 0 & 0 & 0 & 0 & 0 &		0 & 0 & 0 & 0
		\end{pmatrix}
\end{align*}

An Stelle 4 ist in $z$ ein Fehler aufgetreten $\Leftrightarrow |\lbrace j: <v_j,z> = 1\rbrace| > |\lbrace j:<v_j,z> = 0 \rbrace|$
\begin{align*}
	<v_0,z> &= 1 \\
	<v_1,z> &= 0 \\
	<v_2,z> &= 0 \\
	<v_3,z> &= 1 \\
	<v_4,z> &= 1 \\
	<v_5,z> &= 1 
\end{align*}
Es ist also ein Fehler an Position 4 aufgetreten.
\end{myList}
\end{document}
