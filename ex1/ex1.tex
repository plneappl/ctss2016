\documentclass{article}

\usepackage{etex}
\usepackage[utf8]{inputenc}
\usepackage{csquotes}
\usepackage{ngerman}
\usepackage{graphicx}
\usepackage{amssymb}
\usepackage{fancyhdr}
\usepackage{caption}
\usepackage{url}
\usepackage{lastpage}
\usepackage{geometry}
\usepackage{amsmath}
\usepackage{titlesec}
\usepackage{amsthm}
\usepackage{mathtools}
\usepackage{extarrows}
\usepackage{listings} \lstset{numbers=left, numberstyle=\tiny, numbersep=5pt} \lstset{language=Haskell}
\usepackage{stmaryrd}
%Felix: There is a conflict between mathtools and xfraf whem I'm compiling the
% document.
% https://lists.debian.org/debian-tex-maint/2011/03/msg00101.html
% \usepackage{xfrac}
\usepackage{tikz}
\usepackage{epstopdf}
\usepackage{float}
\usepackage{stmaryrd}
\usepackage{centernot}
\usepackage{amssymb}
\usepackage{calc}
\usepackage[nomessages]{fp}
\usepackage{polynom}
\usepackage{ctable}
\usepackage[sharp]{easylist}
\usepackage{siunitx}
\usepackage{pdfpages}
\usepackage{enumerate}
\usepackage{algorithm}
\usepackage{algpseudocode}
\usepackage{pifont}
\usepackage{hyperref}
\usepackage{placeins}
\usepackage[makeroom]{cancel}
%\usepackage{sagetex}

\geometry{a4paper,left=3cm, right=3cm, top=3cm, bottom=3cm}
\pagestyle{fancy}
\fancyhead[C]{Codierungstheorie}
\fancyhead[R]{\today}
\fancyfoot[L]{Simon Wegendt}
\fancyfoot[C]{Seite \thepage /\pageref{LastPage}}
\fancyfoot[R]{David Binder}
\renewcommand{\headrulewidth}{0.4pt}
\renewcommand{\footrulewidth}{0.4pt}
\newcommand{\cmark}{\text{\ding{51}}}%
\newcommand{\xmark}{\text{\ding{55}}}%
\parindent0pt
%\newcommand{\N}{\mbox{$I\!\!N$}}
%\newcommand{\Z}{\mbox{$Z\!\!\!Z$}}
%\newcommand{\Q}{\mbox{$I\:\!\!\!\!\!Q$}}
%\newcommand{\R}{\mbox{$I\!\!R$}}
\newcommand{\N}{\mathbb{N}}
\newcommand{\Z}{\mathbb{Z}}
\newcommand{\Q}{\mathbb{Q}}
\newcommand{\R}{\mathbb{R}}
\newcommand{\C}{\mathcal{C}}
\newcommand{\D}{\mathbb{D}}
\renewcommand{\L}{\mathbb{L}}
\renewcommand{\P}{\mathcal{P}}
\newcommand{\set}[1]{\left\lbrace #1 \right\rbrace}
\newcommand{\tuple}[1]{\left( #1 \right)}
\newcommand{\entspr}{\ensuremath \widehat{=}}
\newcommand{\innervect}[1]{\begin{array}{c}#1\end{array}}
\newcommand{\vect}[1]{\tuple{\innervect{#1}}}
\makeatletter
\newcommand*{\rom}[1]{\expandafter\@slowromancap\romannumeral #1@}
\makeatother
\newcommand{\lgsto}[4]{
  \xrightarrow[
    \overset{
      \text{\scriptsize #3}
    }{
      \text{#4}
    }
  ]{
    \overset{
      \text{\scriptsize #1}
    }{
      \text{#2}
    }
  }
}
\renewcommand{\mod}{\text{ mod }}
\newcommand{\ggT}{\text{ggT}}
\newcommand{\id}{\text{id}}
\newcommand{\IV}{\overset{\text{IV}}{=}}
\newcommand{\ZZ}{\ensuremath{\mathrm{Z\kern-.3em\raise-0.5ex\hbox{Z}}:\phantom{}}}
\newcommand{\grad}{\text{grad}}
\newcommand{\phantomn}{\phantom{}}
\newcommand{\overtext}[2]{\overset{\text{\scriptsize #1}}{#2}}
\newcommand{\Abb}{\text{Abb}}
\newcommand{\seilpmi}{\hspace{3pt}\Longleftarrow\hspace{3pt}}
\newcommand{\Kern}{\text{Kern}}
\newcommand{\Bild}{\text{Bild}}
\newcommand{\Spann}[1]{\left\langle#1\right\rangle}
\newcommand{\Rang}{\text{Rang}}
\renewcommand{\phi}{\varphi}
\newcommand{\ceil}[1]{\left\lceil#1\right\rceil}
\newcommand{\floor}[1]{\left\lfloor#1\right\rfloor}
\newcommand{\TM}[1]{\left\langle#1\right\rangle}
\newcommand{\ot}{\reflectbox{$\to$}}
\newcommand{\underscore}{\underline{\hspace{5pt}}}
\newcommand{\VR}{\text{VR}}

\DeclareMathOperator*{\argmax}{argmax\,}
\DeclareMathOperator*{\argmin}{argmin\,}

\def\splitstring#1#2{%
    \noexpandarg
    \IfSubStr{#1}{#2}{
    \StrBefore{#1}{#2}
    }{#1}
}
\newcommand{\firstline}[1]{\splitstring{#1}{\\}}

\newcommand{\listSettings}{\ListProperties(Numbers1=l, Mark1={)}, Style1*=\bfseries\large, Numbers2=a, Numbers3=R, Numbers4=r, Hide3=2, Mark2=., Style2*={}, Progressive=1em)}

\newcommand{\autobox}[1]{\parbox{\widthof{\firstline{#1}}}{#1}}
\newenvironment{myList}{
  \begin{easylist}[enumerate]\listSettings
}{\end{easylist}}

\newcommand*{\independent}{\ensuremath{\bot\hspace{-0.5em}\bot}}
\newcommand*{\given}{\,|\,}


\fancyhead[L]{Übungsblatt 1}
\begin{document}
\section*{Aufgabe 1}
Wir modellieren die Fehler als Addition von Zahlen aus der Menge $\lbrace 0 \ldots 96\rbrace$ sodass das Ergebnis genau der Veränderung der Ziffer modulo 97 entspricht.
Es wird also das gesendete Zeichen $x$ zu dem empfangenen Zeichen $x' = (x + c) \mod 10$ mit $c\in[0, 9]$.
Beispiele:
\begin{enumerate}
	\item Wird $x = 5$ abgeändert zu $x' = 6$ so wird $x'$  modelliert als $(x + c) \mod 10$, wobei $c$ hier 1 ist.
	\item Wird $x = 5$ abgeändert zu $x' = 4$ so wird $x'$  modelliert als $(x + c) \mod 10$, wobei $c$ hier 9 ist.
\end{enumerate}

Es sei im folgenden $w$ das unmodifizierte Wort, $w'$ das modifizierte Wort und $PS(w)$ die Prüfsumme des Wortes $w$.

\begin{myList}
#
Seien die beiden Positionen, an denen verändert wird, $a$ und $a+1$; d.h. $c_i=0\,\forall i\not= a, i\not=a+1$:
\begin{align*}
	PS(w) &= PS(w') = 1 \\
	 &\Leftrightarrow \sum_{i = 0}^{23} 10^i \cdot (x_i + c_i) \equiv 1 (\mod 97) \\
	  &\Leftrightarrow \left(\sum_{i = 0}^{23} 10^i \cdot x_i \right) + c_a \cdot 10^a + c_{a+1} \cdot 10^{a+1} \equiv 1 (\mod 97) \\
	  &\Leftrightarrow c_a \cdot 10^a + c_{a+1} \cdot 10^{a+1} \equiv 0 (\mod 97) \\
	  &\Leftrightarrow 10^a (c_a + c_{a+1} \cdot 10) \equiv 0 (\mod 97) \\
	  &\Rightarrow (c_a + c_{a+1} \cdot 10) \equiv 0 (\mod 97)
\end{align*}
Wenn also  $(c_a + c_{a+1} \cdot 10) \not\equiv 0 (\mod 97)$, dann wird der Fehler erkannt. Dies ist genau dann der Fall, wenn $(c_a + c_{a+1} \cdot 10) \not= 97$, also $c_a\not=7$ und $c_{a+1}\not=9$.

#
Seien die beiden Positionen, an denen verändert wird, $a$ und $a+2$:
\begin{align*}
	PS(w) &= PS(w') = 1 \\
	&\Leftrightarrow \sum_{i = 0}^{23} 10^i \cdot (x_i + c_i) \equiv 1 (\mod 97) \\
	&\Leftrightarrow \left(\sum_{i = 0}^{23} 10^i \cdot x_i \right) + c_a \cdot 10^a + c_{a+2} \cdot 10^{a+2} \equiv 1 (\mod 97) \\
	  &\Leftrightarrow c_a \cdot 10^a + c_{a+2} \cdot 10^{a+2} \equiv 0 (\mod 97) \\
	  &\Leftrightarrow 10^a (c_a + c_{a+2} \cdot 100) \equiv 0 (\mod 97) \\
	  &\Leftrightarrow 10^a (c_a + c_{a+2} \cdot 3) \equiv 0 (\mod 97) \\
	  &\Rightarrow (c_a + c_{a+2} \cdot 3) \equiv 0 (\mod 97)
\end{align*}

Wenn also $(c_a + c_{a+2} \cdot 3) \not\equiv 0 (\mod 97)$, dann wird der Fehler erkannt. Mit allen Einschränkungen gibt es keine $(c_a, c_{a+2})\not=(0, 0)$, so dass ein solcher Fehler nicht erkannt wird.

#
Sei $w = (x_{23}, \ldots , x_0)$ und $w' = (x_0, x_{23}, \ldots, x_1)$.
Sei $w \in \mathcal{C}$.
\begin{align*}
	&\sum_{i=0}^{23} x_i \cdot 10^i \mod 97 \\
	&= \left(\sum_{i=1}^{23} x_i \cdot 10^i\right) + x_0  \mod 97\\
	&= 10 \cdot \left(\sum_{i=1}^{23} x_i \cdot 10^{i-1}\right) + x_0 \mod 97 \\
	&= 10 \cdot \left(\sum_{i=1}^{23} x_i \cdot 10^{i-1}\right) + 10^{23}x_0 - (10^{23} -1) x_0 \mod 97 \\
	&= \left(\sum_{i=1}^{23} x_i \cdot 10^{i-1}\right) + 10^{23}x_0 - (10^{23} -1) x_0 + 9 \cdot \left(\sum_{i=1}^{23} x_i \cdot 10^{i-1}\right) \mod 97 \\
	&= \left(\sum_{i=1}^{23} x_i \cdot 10^{i-1}\right) + 10^{23}x_0 - 55 x_0 + 9 \cdot \left(\sum_{i=1}^{23} x_i \cdot 10^{i-1}\right) \mod 97
\end{align*}
Also ist $w'$ in $\C$ genau dann wenn $- 55 x_0 + 9 \cdot \left(\sum_{i=1}^{23} x_i \cdot 10^{i-1}\right) \equiv 0 (\mod 97)$
\end{myList}
\section*{Aufgabe 2}
Gegenbeispiel:\\
Sei $\C = \set{000}\subset\set{0, 1}^3$\\
$\implies d(\C)=0$ (per Definition), $n=3$.\\
Setze $t=1\in\N$: $t\leq n$ und $2t+1=3>d(\C)$.\\
Für alle $y\in \set{0, 1}^3$ wird $y$ auf $x=000$ decodiert, also auch für alle $y$ mit $d(x, y)=t$. Da es keine $x'\in\C$ gibt mit $x\not= x'$, ist die Aussage wiederlegt.

\section*{Aufgabe 3}
\begin{myList}
# $\C \subseteq\set{0,1}^7, |C|=3$\\
\ZZ: $d(C)< 5$. Beweis durch Wiederspruch:\\\\
Sei $\C=\set{a=\tuple{a_1, a_2,\ldots, a_7}, b=\tuple{b_1, \ldots}, c=\ldots}$. Angenommen $d(\C)\geq 5$. Dann gilt:\\
$d(a, b)\geq 5, d(b, c)\geq 5, d(c, a)\geq 5$.\\
OBdA seien $a_1\not=b_1, a_2\not=b_2, \ldots, a_5\not= b_5$. Damit $d(b, c)\geq 5$ müssen mindestens 3 der von $a$ unterschiedlichen Buchstaben in $b$ von $c$ unterschiedlich sein, d.h.
$$\exists i_1, i_2, i_3 \leq 5: i_1\not=i_2\not=i_3\not=i_1, b_{i_1}\not= c_{i_1},   b_{i_2}\not= c_{i_2},   b_{i_3}\not= c_{i_3} $$
Da $\C\subseteq\set{0, 1}^7$ ist, müssen gerade diese unterschiedlichen Buchstaben gleich sein wie die jeweiligen aus $a$. Dann kann sich $c$ aber nur noch an höchstens $7-3=4$ Stellen von $a$ unterscheiden. \\
Also ist $d(\C)< 5$.

# Zu konstruieren: ein Blockcode $\C=\set{c_1, c_2, c_3, c_4}\subseteq\set{0,1}^8$ mit $|\C|=4, d(\C)=5$.   \\
Fange an mit $$c_1=0000\;0000$$.\\
$c_2$ muss sich an 5 Stellen unterscheiden: $$c_2= \mbox{*}\mbox{*}\mbox{*}1\;1111$$\\
$c_3$ muss sich an 5 Stellen von $c_1$ unterscheiden: $$c_3=11\mbox{*}1\;\mbox{*}1\mbox{*}1$$
$c_2$ muss sich an 5 Stellen von $c_3$ unterscheiden: $$c_2=00\mbox{*}1\;1111$$
$$c_3=11\mbox{*}1\;0101$$
$c_4$ muss sich an 5 Stellen von $c_3$ unterscheiden:
$$c_4=\mbox{*}\mbox{*}\mbox{*}0\;1010$$
$c_4$ muss sich an 5 Stellen von $c_1$ unterscheiden:
$$c_4=1110\;1010$$
$c_2$ und $c_3$ müssen sich an Stelle 3 noch unterscheiden, damit sie sich an 5 Stellen unterscheiden:
$$c_2=0001\;1111, c_3=1111\;0101$$
Finaler code:
$$\C=\begin{pmatrix}
0000&0000\\
0001&1111\\
1111&0101\\
1110&1010
\end{pmatrix}$$

# Es gibt keinen Code wie in b) mit $|\C|=5$, denn $|\C|\overset ! \leq A_2(8, 5) = A_2(9, 6) = 4$ \\(\url{http://www.win.tue.nl/~aeb/codes/binary-1.html}).
\end{myList}

\section*{Aufgabe 4}
Sei $\C$ die Menge aller Palindrome über $\Z_2$ der Länge $n$. \\
\begin{myList}
# Zu bestimmen ist $|\C|$. Wir unterscheiden zunächst zwei Fälle:
## $n$ gerade, d.h. es gibt ein $k\in\N$, so dass $n=2k$. Dann ist $\C=\set{x\,x^r|x\in\Z^k_2}$.
$$|\C| = |\Z^k_2|=2^k=2^{\frac n 2}=2^{\ceil{\frac n 2}}$$
## $n$ ungerade, d.h. es gibt ein $k\in\N$, so dass $n=2k+1$. Dann ist $\C=\set{x\,y\,x^r|x\in\Z^k_2, y\in\Z_2}$.
$$|\C| = |\Z^k_2|\cdot|\Z_2|=2^k\cdot 2=2^{k+1}=2^{\ceil{\frac n 2}}$$
Also gilt, unabhängig von der Parität von $n$:
$$|\C|=2^{\ceil{\frac n 2}}$$
# Zu bestimmen ist $d(\C)$. Wir unterscheiden wieder, ob $n$ gerade ist:
## Falls $n$ gerade ist, d.h. $n=2k$, nennen wir zu $x\in\Z_2^k$ ein Wort $x'$ in $\Z_2^k$, so dass $d(x, x')=1$ (z.B. $x + 1\pmod {2^k}$). Es gilt:
### Für $x\in\Z_2^k$ sind $x_1=x\,x^r$ und $x_2=x'\,x'^r$ in $\C$. $d(x_1, x_2)=d(x, x')+d(x, x')=2$. Also gilt:
$$d(\C)\leq 2$$
### Es gibt keine zwei Worte $x_1, x_2$ in $\C$ mit $d(x_1, x_2)=1$, denn sonst wäre eins von beiden kein Palindrom: \\
Seien $x_1=x_a\,x_b,~x_2=x_c\, x_d$ beide in $\C, |x_a|=|x_c|=k,~x_a=x_b^r,~x_c=x_d^r$. Ist nun $d(x_1, x_2)=1$, muss entweder $$d(x_a, x_c)=1, d(x_b, x_d)=0$$ oder $$d(x_a, x_c)=0, d(x_b, x_d)=1$$ gelten.
Da $x_a=x_b^r$ und $x_c=x_d^r$ ist, muss aber auch gelten:
$$d(x_a, x_c)=d(x_b, x_d)$$
Dies ist ein Wiederspruch zu dem `Entweder oder'. Also gilt:
$$d(\C)\geq 2$$
\\Dann muss $d(\C)$ also genau
$$d(\C)=2=2-n\pmod 2$$
## Falls $n$ ungerade ist, d.h. $n=2k+1$, sind $x_1=x\,0\,x^r$ und $x_2=x\,1\,x^r$ in $\C$ für $x\in\Z_2^k$ beliebig. Dann gilt: $$d(x_1, x_2)=0+1+0=1=2-n\pmod 2$$. Da $\C$ eine Menge ist, es also keine doppelten Worte gibt, ist dies gerade der Minimalabstand $d(\C)$.\\\\
Also gilt, unabhängig von der Parität von $n$:
$$d(\C)=2-n\pmod 2$$
\end{myList}

\section*{Aufgabe 5}
$\mathcal{C} = \lbrace (0,0,0,0,0) , (0,1,1,0,1) , (1,0,1,1,0) , (1,1,0,1,1) \rbrace$

Für $\mathcal{C}$ gelten folgende allgemeine Behauptungen.
\begin{itemize}
	\item Für jedes Codewort $w$ aus $\mathcal{C}$ existieren 2 Codewörter mit Hamming-Abstand 3 und 1 Codewort mit Hamming-Abstand 4 zu $w$.
	Die beiden Codewörter mit Abstand 3 zu $w$ haben untereinander einen Abstand von $4$.
	\item Für jedes Codewort $w$ und jede Position $i$ gilt: Es existieren 2 andere Codewörter die an Position $i$ nicht mit $w$ übereinstimmen und 1 Codewort das an Position $i$ mit $w$ übereinstimmt.
	\item Im folgenden steht $w$ für das gesendete Codewort und $w'$ für das empfangene, modifizierte Wort.
\end{itemize}
\begin{myList}
#
Es gibt genau 10 Möglichkeiten dass zwei Fehler auftreten da $10 = \binom{5}{2}$, und $\binom{5}{2}$ die Anzahl der Möglichkeiten angibt 2 Positionen aus 5 möglichen auszuwählen.\medskip

Möglichkeiten dass falsch dekodiert wird:\\
Die Bedingung kann ausgeschrieben werden als: $\exists v \in \mathcal{C}: d(v,w') < 2$.
Dafür kommen prinzipiell nur die beiden Codewörter mit Hammingabstand 3 zu $w$ in Betracht.
Für jede der beiden in Frage kommenden Codewörter gibt es $\binom{3}{2} = 3$ Möglichkeiten.
Diese jeweils 3 Möglichkeiten sind disjunkt da die beiden Codewörter einen Abstand von 4 zueinander haben und daher in maximal einer Position übereinstimmen können.
Also existieren 6 Möglichkeiten dass falsch dekodiert wird.
\medskip

Möglichkeiten dass uneindeutig dekodiert wird:\\
Die Bedingung kann ausgeschrieben werden als: $\exists v \in \mathcal{C}: d(v,w') = 2 \wedge \neg \exists v' \in \mathcal{C}: d(v',w') < 2$
Für den ersten Teil der Bedingung kommt nur das Codewort mit Hamming-Abstand 4 zu $w$ in Betracht.
Dafür existieren $\binom{4}{2} = 6$ Möglichkeiten.
Von diesen 6 Möglichkeiten entfallen 2 bei denen eindeutig falsch kodiert wird (zweiter Teil der Bedingung).
Die Anzahl ist hier 2 da die beiden Codewörter mit Hamming-Abstand 3 zu $w$ an keiner Stelle übereinstimmen ausser an der Stelle an der das Codewort mit Hamming-Abstand 4 zu $w$ sich von $w$ unterscheidet.

#
3 Fehler:\\
Es gibt $\binom{5}{3} = 10$ Möglichkeiten wie Fehler auftreten können.
Es gibt immer ein Codewort $v$ mit $d(v,w') < 3$.
Beweis:\\
Jede Modifikation die für eines der drei anderen Codewörter aus $\mathcal{C}$ den Abstand zu $w'$ erhöht verringert den Abstand von $w'$ zu den anderen beiden Codewörtern.
Sind zB 2 aus 3 Modifikationen negativ für $v'$, so ist dieselbe Modifikation positiv für die beiden anderen Codewörter $v''$ und $v'''$.
\medskip

4 Fehler:\\
Es gibt $\binom{5}{4} = 5$ Möglichkeiten wie Fehler auftreten können.
Sei $w$ das gesendete Codewort, $w'$ das modifizierte Wort und $v$ das Codewort welches zu $w$ den Hamming-Abstand 4 hat.
Dann gilt dass $d(w',v) = 0$ oder $d(w',v) = 2$ (da die nicht modifizierte Stelle entweder mit $v$ übereinstimmen kann oder nicht).
Damit existiert immer ein Codewort welches einen geringeren Hamming-Abstand von $w'$ als $w$ hat.
\medskip

5 Fehler:\\
Es gibt $\binom{5}{5} = 1$ Möglichkeit wie Fehler auftreten können.
Das dadurch entstehende Wort hat den Maximalabstand 5 vom Ursprungswort, und da keine 3 Wörter jeweils den Maximalabstand voneinander haben können muss immer ein Codewort mit kleinerem Hamming-Abstand existieren.
Es wird also immer falsch dekodiert.
#
Sei $p_i$ die Wahrscheinlichkeit dass $i$ Fehler auftreten. Es gilt:
\begin{equation*}
	p_i = p^i \cdot (1-p)^{5-i}
\end{equation*}
Sei $f_i$ die Wahrscheinlichkeit dass bei $i$ aufgetretenen Fehlern falsch dekodiert wird. Aus den vorherigen Aufgabenteilen und für die Trivialfälle $i = 0$ und $i = 1$ gilt:
\begin{equation*}
	f_0 = f_1 = 0 \qquad f_2 = 0.4\cdot 0.5 + 0.6 \qquad f_3 = 1 \qquad f_4 = 1 \qquad f_5 = 1
\end{equation*}
Dann ist die Decodierfehlerwahrscheinlichkeit gegeben durch:
\begin{align*}
	&p_0 \cdot f_0 + p_1 \cdot f_1 + p_2 \cdot f_2 + p_3 \cdot f_3 + p_4 \cdot f_4 + p_5 \cdot f_5 \\
	&= p_2 \cdot 0.8 + p_3  + p_4  + p_5 \\
	&= (p^2 \cdot (1-p)^3) \cdot 0.8 + (p^3 \cdot (1-p))^2) + (p^4 \cdot (1-p)^1) + p^5
\end{align*}
\end{myList}
\end{document}





