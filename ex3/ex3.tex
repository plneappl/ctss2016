\documentclass{article}

\usepackage{etex}
\usepackage[utf8]{inputenc}
\usepackage{csquotes}
\usepackage{ngerman}
\usepackage{graphicx}
\usepackage{amssymb}
\usepackage{fancyhdr}
\usepackage{caption}
\usepackage{url}
\usepackage{lastpage}
\usepackage{geometry}
\usepackage{amsmath}
\usepackage{titlesec}
\usepackage{amsthm}
\usepackage{mathtools}
\usepackage{extarrows}
\usepackage{listings} \lstset{numbers=left, numberstyle=\tiny, numbersep=5pt} \lstset{language=Haskell}
\usepackage{stmaryrd}
%Felix: There is a conflict between mathtools and xfraf whem I'm compiling the
% document.
% https://lists.debian.org/debian-tex-maint/2011/03/msg00101.html
% \usepackage{xfrac}
\usepackage{tikz}
\usepackage{epstopdf}
\usepackage{float}
\usepackage{stmaryrd}
\usepackage{centernot}
\usepackage{amssymb}
\usepackage{calc}
\usepackage[nomessages]{fp}
\usepackage{polynom}
\usepackage{ctable}
\usepackage[sharp]{easylist}
\usepackage{siunitx}
\usepackage{pdfpages}
\usepackage{enumerate}
\usepackage{algorithm}
\usepackage{algpseudocode}
\usepackage{pifont}
\usepackage{hyperref}
\usepackage{placeins}
\usepackage[makeroom]{cancel}
%\usepackage{sagetex}

\geometry{a4paper,left=3cm, right=3cm, top=3cm, bottom=3cm}
\pagestyle{fancy}
\fancyhead[C]{Codierungstheorie}
\fancyhead[R]{\today}
\fancyfoot[L]{Simon Wegendt}
\fancyfoot[C]{Seite \thepage /\pageref{LastPage}}
\fancyfoot[R]{David Binder}
\renewcommand{\headrulewidth}{0.4pt}
\renewcommand{\footrulewidth}{0.4pt}
\newcommand{\cmark}{\text{\ding{51}}}%
\newcommand{\xmark}{\text{\ding{55}}}%
\parindent0pt
%\newcommand{\N}{\mbox{$I\!\!N$}}
%\newcommand{\Z}{\mbox{$Z\!\!\!Z$}}
%\newcommand{\Q}{\mbox{$I\:\!\!\!\!\!Q$}}
%\newcommand{\R}{\mbox{$I\!\!R$}}
\newcommand{\N}{\mathbb{N}}
\newcommand{\Z}{\mathbb{Z}}
\newcommand{\Q}{\mathbb{Q}}
\newcommand{\R}{\mathbb{R}}
\newcommand{\C}{\mathcal{C}}
\newcommand{\D}{\mathbb{D}}
\renewcommand{\L}{\mathbb{L}}
\renewcommand{\P}{\mathcal{P}}
\newcommand{\set}[1]{\left\lbrace #1 \right\rbrace}
\newcommand{\tuple}[1]{\left( #1 \right)}
\newcommand{\entspr}{\ensuremath \widehat{=}}
\newcommand{\innervect}[1]{\begin{array}{c}#1\end{array}}
\newcommand{\vect}[1]{\tuple{\innervect{#1}}}
\makeatletter
\newcommand*{\rom}[1]{\expandafter\@slowromancap\romannumeral #1@}
\makeatother
\newcommand{\lgsto}[4]{
  \xrightarrow[
    \overset{
      \text{\scriptsize #3}
    }{
      \text{#4}
    }
  ]{
    \overset{
      \text{\scriptsize #1}
    }{
      \text{#2}
    }
  }
}
\renewcommand{\mod}{\text{ mod }}
\newcommand{\ggT}{\text{ggT}}
\newcommand{\id}{\text{id}}
\newcommand{\IV}{\overset{\text{IV}}{=}}
\newcommand{\ZZ}{\ensuremath{\mathrm{Z\kern-.3em\raise-0.5ex\hbox{Z}}:\phantom{}}}
\newcommand{\grad}{\text{grad}}
\newcommand{\phantomn}{\phantom{}}
\newcommand{\overtext}[2]{\overset{\text{\scriptsize #1}}{#2}}
\newcommand{\Abb}{\text{Abb}}
\newcommand{\seilpmi}{\hspace{3pt}\Longleftarrow\hspace{3pt}}
\newcommand{\Kern}{\text{Kern}}
\newcommand{\Bild}{\text{Bild}}
\newcommand{\Spann}[1]{\left\langle#1\right\rangle}
\newcommand{\Rang}{\text{Rang}}
\renewcommand{\phi}{\varphi}
\newcommand{\ceil}[1]{\left\lceil#1\right\rceil}
\newcommand{\floor}[1]{\left\lfloor#1\right\rfloor}
\newcommand{\TM}[1]{\left\langle#1\right\rangle}
\newcommand{\ot}{\reflectbox{$\to$}}
\newcommand{\underscore}{\underline{\hspace{5pt}}}
\newcommand{\VR}{\text{VR}}

\DeclareMathOperator*{\argmax}{argmax\,}
\DeclareMathOperator*{\argmin}{argmin\,}

\def\splitstring#1#2{%
    \noexpandarg
    \IfSubStr{#1}{#2}{
    \StrBefore{#1}{#2}
    }{#1}
}
\newcommand{\firstline}[1]{\splitstring{#1}{\\}}

\newcommand{\listSettings}{\ListProperties(Numbers1=l, Mark1={)}, Style1*=\bfseries\large, Numbers2=a, Numbers3=R, Numbers4=r, Hide3=2, Mark2=., Style2*={}, Progressive=1em)}

\newcommand{\autobox}[1]{\parbox{\widthof{\firstline{#1}}}{#1}}
\newenvironment{myList}{
  \begin{easylist}[enumerate]\listSettings
}{\end{easylist}}

\newcommand*{\independent}{\ensuremath{\bot\hspace{-0.5em}\bot}}
\newcommand*{\given}{\,|\,}


\fancyhead[L]{Übungsblatt 3}
\setcounter{MaxMatrixCols}{20}
\begin{document}
\section*{Aufgabe 11}
Gegeben sei ein binär symmetrischer Kanal mit Fehlerwahrscheinlichkeit $p = \frac{1}{4}$
\begin{myList}
#
Berechnen Sie die Kapazität $C$ des Kanals.

Nach Satz 2.22 ist die Kapazität eines binär symmetrischen Kanals mit Fehlerwahrscheinlichkeit $p$ genau:
\begin{align*}
	C &= 1 - H_p = 1 + p \cdot \log p + (1-p)\cdot \log (1-p) \\
	&= 1 + \frac{1}{4} \cdot \log \left(\frac{1}{4}\right) + \frac{3}{4}\cdot \log \left(\frac{3}{4}\right) \\
	&= \frac{3}{4} \log(3) - 1 \approx  0.189
\end{align*} 
#
Bestimmen Sie $k,n \in \N$ so dass ein Code $\C$ der Länge $n$ über $\lbrace 0,1 \rbrace$ mit Rate $R = \frac{k}{n}$ und folgenden Eigenschaften existiert: \ldots

\underline{Bestimme $\beta$:}

Aus der Bedingung $ H_{p+\beta} - H_{p} < \frac{0.01}{4} = \frac{1}{400 }$, wenn $p + \beta$ durch $\lambda$ abgekürzt wird, folgt:
\begin{align*}
	H_\lambda - H_p &< \frac{0.01}{4} = \frac{1}{400} \\
	\Leftrightarrow H_\lambda &< \frac{1}{400} + 2 - \frac{3}{4}\log 3 \\
	\Leftrightarrow \lambda \log \frac{1}{\lambda} + (1-\lambda) \log(\frac{1}{1-\lambda}) &< \frac{1}{400} + 2 - \frac{3}{4}\log 3 \\
	\Leftrightarrow \lambda &< 0.251583 \\
	\Leftrightarrow \beta &< 0.001583
\end{align*}
Um n später möglichst klein wählen zu können sollte $\beta$ möglichst groß gewählt werden.
Wir wählen also $\beta = 0.00158$.
Dieses $\beta$ erfüllt natürlich auch die andere Ungleichung $\beta \leq \frac{1}{2} - p = 0.25$.

\underline{Bestimme $n$:}

Aus der Bedingung $2^{\frac{n \epsilon'}{4} > \frac{2}{\epsilon}}$ folgt:
\begin{align*}
	2^{\frac{n \epsilon'}{4}} &> \frac{2}{\epsilon} \\
	\Leftrightarrow \frac{n \epsilon'}{4} &> \log \left( \frac{2}{\epsilon}\right)\\
	\Leftrightarrow n &> \frac{\log\left( \frac{2}{\epsilon}\right) \cdot 4}{\epsilon'} = \log(200) \cdot 4 \cdot 100 \approx 3057 \\
	\Leftarrow n &> 3058
\end{align*}

Aus der Bedingung $n > \frac{2p(1-p)}{\epsilon \cdot \beta^2}$ folgt:
\begin{align*}
	n &> \frac{2p(1-p)}{\epsilon \cdot \beta^2}\\
	\Leftrightarrow n &> 37.5 \cdot \frac{1}{\beta^2}
\end{align*}
Für $\beta = 0.00158$ ergibt sich $\approx 1.5 \cdot 10^7$.
Wir wählen also $n = 1.6\cdot 10^7$.
(Hier könnte man $n$ natürlich noch genauer bestimmen.)

\underline{Bestimme $k$:}

Für die Bedingung $C - \epsilon < \frac{k}{n} < C - \frac{\epsilon}{2}$ erhalten wir, wenn wir die Werte für $n, \epsilon$ und $C$ einsetzen:
\begin{align*}
	0,179 < \frac{k}{1.6\cdot 10^7} < 0.184 \\
	\Leftrightarrow 0.179 \cdot 1.6\cdot 10^7 < k < 0.184 \cdot 1.6\cdot 10^7
\end{align*}
Wir wählen z.B. $k = 0.18 \cdot 1.6 \cdot 10^7 = 2 880 000$.
\end{myList}

\section*{Aufgabe 12}
Sei $\C$ der binäre lineare Code mit der Erzeugermatrix
\begin{equation*}
	G = \begin{pmatrix}
		1 & 1 & 0 & 0 & 1 & 0 \\
		1 & 0 & 1 & 1 & 1 & 0 \\
		1 & 1 & 1 & 0 & 0 & 1
	\end{pmatrix}
\end{equation*}
\begin{myList}
#
Bestimmen Sie eine Kontrollmatrix für $\C$.\\\\
\begin{align*}
G=&\begin{pmatrix}
1&1&0&0&1&0\\
1&0&1&1&1&0\\
1&1&1&0&0&1
\end{pmatrix}\lgsto{\rom 1 + \rom 3}{\rom 2}{\rom 1}{}\begin{pmatrix}
0&0&1&0&1&1\\
1&0&1&1&1&0\\
1&1&0&0&1&0
\end{pmatrix}\lgsto{\rom 2 + \rom 1}{\rom 3}{\rom 1}{}\begin{pmatrix}
1&0&0&1&0&1\\
1&1&0&0&1&0\\
0&0&1&0&1&1
\end{pmatrix}\\\lgsto{\rom 1}{\rom 2 + \rom 1}{\rom 3}{}&\begin{pmatrix}
1&0&0&1&0&1\\
0&1&0&1&1&1\\
0&0&1&0&1&1
\end{pmatrix}\\
\implies H=&\begin{pmatrix}
1&1&0&1&0&0\\
0&1&1&0&1&0\\
1&1&1&0&0&1
\end{pmatrix}
\end{align*}
#
Bestimmen Sie den Minimalabstand von $\C$.\\\\
$d(\C)=max\set{i\in\N|\text{je $i-1$ Spalten in $H$ sind linear unabhängig}}=3$
\end{myList}

\section*{Aufgabe 13}
\begin{myList}
#
Konstruieren Sie eine Kontrollmatrix und eine Erzeugermatrix für einen $[15,11]$-Hamming-Code $\C$.\\\\
Bestimme $q$ und $l$:
\begin{align*}
&n=15, k=11=n-l\\
\implies& l = 4\\
&n=\frac{q^l-1}{q-1}=15 = \frac{q^4-1}{q-1} \\
&\iff q^4 - 15q + 14 = 0\\
q=2:&~ 2^4 - 30 + 14 = 30-30=0 \checkmark
\end{align*}
. Wähle zu jedem 1-dimensionalem Unterraum von $Z_2^4$ einen Vektor $\not= \mathbf 0$ und schreibe diese als Spaltenvektoren in die Prüfmatrix:\\
$$H=\begin{pmatrix}
1&0&0&0&1&1&1&0&0&0&1&1&1&1&1\\
0&1&0&0&1&0&0&1&1&0&1&0&1&1&1\\
0&0&1&0&0&1&0&1&0&1&1&1&0&1&1\\
0&0&0&1&0&0&1&0&1&1&0&1&1&0&1
\end{pmatrix}$$
Da $H$ von der Form
$$\begin{pmatrix}
I_4 & M_1
\end{pmatrix}$$
ist, kann die Erzeugermatrix direkt aufgeschrieben werden als
$$G=\begin{pmatrix}
 M_1^T & -I_{11}
\end{pmatrix}\overset{\Z_2}=\begin{pmatrix}
 M_1^T & I_{11}
\end{pmatrix}$$
$$=\begin{pmatrix}
1&1&0&0&1&0&0&0&0&0&0&0&0&0&0\\
1&0&1&0&0&1&0&0&0&0&0&0&0&0&0\\
1&0&0&1&0&0&1&0&0&0&0&0&0&0&0\\
0&1&1&0&0&0&0&1&0&0&0&0&0&0&0\\
0&1&0&1&0&0&0&0&1&0&0&0&0&0&0\\
0&0&1&1&0&0&0&0&0&1&0&0&0&0&0\\
1&1&1&0&0&0&0&0&0&0&1&0&0&0&0\\
1&0&1&1&0&0&0&0&0&0&0&1&0&0&0\\
1&1&0&1&0&0&0&0&0&0&0&0&1&0&0\\
1&1&1&0&0&0&0&0&0&0&0&0&0&1&0\\
1&1&1&1&0&0&0&0&0&0&0&0&0&0&1
\end{pmatrix}$$
#
Codieren Sie das Infowort $(0,1,0,1,0,1,0,1,0,1,0)$ mit der Erzeugermatrix aus a).\\\\
\begin{align*}
&\phantom{\phantom{}=\phantom{}}\begin{pmatrix}0&1&0&1&0&1&0&1&0&1&0\end{pmatrix}\cdot G\\
&\overset{*^1}=\begin{pmatrix}1&0&1&0&0&1&0&1&0&1&0&1&0&1&0\end{pmatrix}
\end{align*}
{\small $*^1$ mit python/numpy ausgerechnet}
#
Decodieren Sie das empfangene Wort $w=(0,0,0,0,0,1,0,0,0,1,0,0,0,0,0)$.\\\\
\begin{align*}
&\phantom{\phantom{}=\phantom{}}H\cdot w^T = H\cdot \begin{pmatrix}0\\0\\0\\0\\0\\1\\0\\0\\0\\1\\0\\0\\0\\0\\0\end{pmatrix}\overset{*^1}=\begin{pmatrix}1\\0\\0\\1\end{pmatrix}
\end{align*}
Da $Hw^T\not=\mathbf 0$, ist $w$ kein Codewort. Es gibt 15 Wörter $w_1\ldots, w_{15}$ mit $d(w, w_i)=1$ für alle $i$. Wenn ein $w_i$ davon ein Codewort ist, decodiere $w$ zu $w_i$. Sonst decodiere $w$ zu $\mathbf 0$, da $d(w, \mathbf 0)=2$ und damit minimal ist.\\
Matrix $W=\begin{pmatrix}w_1^T&w_2^T&\ldots&w_{15}^T\end{pmatrix}$:
\begin{align*}
W&=\begin{pmatrix}
0&0&0&0&0&0&0&0&0&0&0&0&0&0&0\\
0&0&0&0&0&0&0&0&0&0&0&0&0&0&0\\
0&0&0&0&0&0&0&0&0&0&0&0&0&0&0\\
0&0&0&0&0&0&0&0&0&0&0&0&0&0&0\\
0&0&0&0&0&0&0&0&0&0&0&0&0&0&0\\
1&1&1&1&1&1&1&1&1&1&1&1&1&1&1\\
0&0&0&0&0&0&0&0&0&0&0&0&0&0&0\\
0&0&0&0&0&0&0&0&0&0&0&0&0&0&0\\
0&0&0&0&0&0&0&0&0&0&0&0&0&0&0\\
1&1&1&1&1&1&1&1&1&1&1&1&1&1&1\\
0&0&0&0&0&0&0&0&0&0&0&0&0&0&0\\
0&0&0&0&0&0&0&0&0&0&0&0&0&0&0\\
0&0&0&0&0&0&0&0&0&0&0&0&0&0&0\\
0&0&0&0&0&0&0&0&0&0&0&0&0&0&0\\
0&0&0&0&0&0&0&0&0&0&0&0&0&0&0
\end{pmatrix}+I_{15}\\
&=\begin{pmatrix}
1&0&0&0&0&0&0&0&0&0&0&0&0&0&0\\
0&1&0&0&0&0&0&0&0&0&0&0&0&0&0\\
0&0&1&0&0&0&0&0&0&0&0&0&0&0&0\\
0&0&0&1&0&0&0&0&0&0&0&0&0&0&0\\
0&0&0&0&1&0&0&0&0&0&0&0&0&0&0\\
1&1&1&1&1&0&1&1&1&1&1&1&1&1&1\\
0&0&0&0&0&0&1&0&0&0&0&0&0&0&0\\
0&0&0&0&0&0&0&1&0&0&0&0&0&0&0\\
0&0&0&0&0&0&0&0&1&0&0&0&0&0&0\\
1&1&1&1&1&1&1&1&1&0&1&1&1&1&1\\
0&0&0&0&0&0&0&0&0&0&1&0&0&0&0\\
0&0&0&0&0&0&0&0&0&0&0&1&0&0&0\\
0&0&0&0&0&0&0&0&0&0&0&0&1&0&0\\
0&0&0&0&0&0&0&0&0&0&0&0&0&1&0\\
0&0&0&0&0&0&0&0&0&0&0&0&0&0&1
\end{pmatrix}\\
H\cdot W&\overset{*^1}=\begin{pmatrix}
0&1&1&1&0&0&0&1&1&1&0&0&0&0&0\\
0&1&0&0&1&0&0&1&1&0&1&0&1&1&1\\
0&0&1&0&0&1&0&1&0&1&1&1&0&1&1\\
1&1&1&0&1&1&0&1&0&0&1&0&0&1&0
\end{pmatrix}
\end{align*}
Da die siebte Spalte der 0-Vektor ist, decodieren wir $w$ zu dem Codewort $$w_7=\begin{pmatrix}0&0&0&0&0&1&1&0&0&1&0&0&0&0&0\end{pmatrix}$$ Das Originalwort $w'$, sodass
$w'\cdot G=w_7$, ist, da $G$ die Einheitsmatrix enthält, leicht abzulesen (die letzten 11 Spalten) als
$$w'=\begin{pmatrix}0&1&1&0&0&1&0&0&0&0&0\end{pmatrix}$$
\end{myList}

\section*{Aufgabe 14}
Der Code $\C \subseteq \Z^{11}_3$ sei durch seine Kontrollmatrix

\begin{equation*}
	H =
	\begin{pmatrix}
	0 & 2 & 1 & 1 & 2 & 2 & 1 & 0 & 0 & 0 & 0 \\
	2 & 0 & 2 & 1 & 1 & 2 & 0 & 1 & 0 & 0 & 0 \\
	1 & 2 & 0 & 2 & 1 & 2 & 0 & 0 & 1 & 0 & 0 \\
	1 & 1 & 2 & 0 & 2 & 2 & 0 & 0 & 0 & 1 & 0 \\
	2 & 1 & 1 & 2 & 0 & 2 & 0 & 0 & 0 & 0 & 1
	\end{pmatrix}
\end{equation*}
gegeben.
(Es handelt sich um den ternären Golay-Code.)
\begin{myList}
#
Bestimmen Sie den Minimalabstand von $\C$.
Wieviele Fehler kann $\C$ korrigieren?

Nach Satz 3.10 gilt $d(\C) = wt(\C) = \min\limits_{r \in \N} \lbrace \text{Es gibt r linear abhängige Spalten in H} \rbrace$

Wir zeigen dass $d(\C) = 5$, müssen also zeigen dass es keine 4-elementige Teilmenge von Spaltenvektoren aus $H$ gibt die linear abhängig ist.

Also hat $\C$ den Minimalabstand 5. Damit kann der Code $2$ Fehler korrigieren.
#
Zeigen Sie, dass $\C$ perfekt ist.

Nach 3.14/3.15 ist $\C$ perfekt falls die folgende Gleichung für $\C$ gilt:
\begin{equation*}
	q^{n-k} = \sum\limits^{t}_{i = 0} \binom{n}{i} (q-1)^i
\end{equation*}
Wobei $t$ maximal sei mit $2t +1 \leq d(\C)$.

Nach Aufgabenteil a) wissen wir dass $d(\C) = 5$, also muss $t =  2$ sein.
Da wir als sonstige Werte haben $q = 3$, $n = 11$, $k = 6$(Da $H$ eine $(n-k) \times n$ Matrix ist) müssen wir zeigen:
\begin{align}
	243 &=  3^5 = \sum\limits_{i = 0}^{2} \binom{11}{i} (3-1)^i \\
	&= \binom{11}{0} \cdot 2^0 + \binom{11}{1} 2^1 + \binom{11}{2} \cdot 2^2 \\
	&= 1 + 22 + 220 = 243
\end{align}
\end{myList}

\section*{Aufgabe 15}
\begin{myList}
#
Sei $\C$ ein linearer MDS-Code der Länge $n$ über $K$, $|K| = q$ mit $d(\C) = 3$.
Zeigen Sie, dass $n \leq q + 1$.
(Hinweis: Kugelpackungsschranke)

Aus $\C$ MDS Code und $d = 3$ wissen wir dass $k = n-d +1 = n-2$.\\
Aus $d(\C) = 3$ folgt $ t = 1$.\\
Setze diese beiden Werte in die Formel der Kugelpackungsschranke ein und erhalte:
\begin{align*}
	\sum\limits_{i = 0}^{1} \binom{n}{i}(q-1)^i &\leq q^2 \\
	\Leftrightarrow 1 + n\cdot (q-1) &\leq q^2 \\
	\Leftrightarrow n &\leq \frac{q^2 -1}{q - 1} = \frac{(q + 1)\cdot (q - 1)}{q - 1} = q + 1
\end{align*}
#
Welche Hamming-Codes sind MDS-Codes?

Sei $\C$ perfekter $[n,k]$-Code mit $d(\C) = 3$ über Alphabet $K$ mit $|K| = q$ und $q$ Primzahlpotenz.\\
Dann gilt:
\begin{itemize}
	\item $\C$ MDS $\Leftrightarrow k = n-2$ (Definition von MDS-Codes)
	\item $\C$ Hamming-Code $\Leftrightarrow n = \frac{q^{n-k}-1}{q-1}$ (Definition Hamming-Code)
	\item Daraus folgt $\C$ Hamming-Code und MDS $\Leftrightarrow n = \frac{q^2 -1}{q-1} = q+1$
\end{itemize}
Also sind exakt die Hamming-Codes MDS-Codes für die gilt $n = q + 1$.
\end{myList}
\section*{Aufgabe 16}
Sei $p$ eine Primzahl. Zeigen Sie, dass für einen linearen perfekten Code $\C$ der Länge 6 über $\Z_p$ gilt:\\
$p=5$ und $\C$ ist ein Hamming-Code über $\Z_5$. Welche Dimension hat dieser Hamming-Code?\\\\
Sei $p$ eine Primzahl, $\C$ ein linearer perfekter Code mit $n=6$ über $\Z_p$. Da $\C$ perfekt ist, gilt mit der Kugelpackungsschranke:
\begin{align*}
&&p^k=|\C|&\overset!=\dfrac{p^n}{\sum\limits_{i=0}^t {n\choose i}(p-1)^i}=\dfrac{p^6}{\sum\limits_{i=0}^t {6\choose i}(p-1)^i}\\
\iff&&p^{6-k}&=\sum\limits_{i=0}^t {6\choose i}(p-1)^i
\intertext{$d\leq n\implies t \leq \frac 5 2\implies t\leq 2$}
\intertext{Fall 1: $t=0$: ($\C=\Z_p^n$)}
&&p^{6-k}&=\sum\limits_{i=0}^t {6\choose i}(p-1)^i=1~\implies k= 6
\intertext{In diesem Fall ist $p$ also frei wählbar. Nicht-triviale Fälle folgen.}
\intertext{Fall 2: $t=1$:}
&&p^{6-k}&=\sum\limits_{i=0}^t {6\choose i}(p-1)^i=6p-6+1=6p-5\\
\iff&&p^{6-k}+5&=6p\\
\iff&&p^{5-k}+\frac 5 p&=6
\intertext{Damit Gleichheit gilt, muss 5 durch $p$ teilbar sein, denn $p^{5-k}$ ist eine natürliche Zahl. 5 hat nur die Teiler 1 und 5, und da $p$ eine Primzahl ist, muss $p=5$ sein. $k$ ist dann:}
&&5^{5-k}+1&=6
\iff&5^{5-k}&=5\iff k=5
\intertext{Fall 3: $t=2$:}
&&p^{6-k}&=\sum\limits_{i=0}^t {6\choose i}(p-1)^i=15p^2-30p+15+6p-6+1=15p^2-24p+10\\
\iff&&p^{6-k}-10&=15p^2-24p\\
\iff&&p^{5-k}-\frac {10} p&=15p-24
\intertext{Damit Gleichheit gilt, muss 10 durch $p$ teilbar sein. Also muss $p\in\set{2, 5}$ sein. Fall 3.1: p=2:}
&&2^{5-k}-5&=30-24=6
&&2^{5-k}&=11~~\lightning
\intertext{Fall 3.2: p=5:}
&&5^{5-k}-2&=15\cdot5-24\\
\iff&&5^{5-k}&=75-22 = 53~~\lightning
\end{align*}
Also wurde gezeigt: Wenn $\C$ nicht trivial ist, d.h. $\C\not=\Z_p^6$, dann ist $p=5$. Außerdem muss $t=1$ sein, und damit $d=2t+1=3$. Die Dimension des Codes ist $k=5$.\\

\end{document}
