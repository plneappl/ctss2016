\documentclass{article}

\usepackage[utf8]{inputenc}
\usepackage{ngerman}
\usepackage{graphicx}
\usepackage{amssymb}
\usepackage{fancyhdr}
\usepackage{caption}
\usepackage{url}
\usepackage{lastpage}
\usepackage{geometry}
\usepackage{amsmath}
\usepackage{titlesec}
\usepackage{amsthm}
\usepackage{mathtools}
\usepackage{extarrows}
\usepackage{listings} \lstset{numbers=left, numberstyle=\tiny, numbersep=5pt} \lstset{language=R}
\usepackage{stmaryrd}
%Felix: There is a conflict between mathtools and xfraf whem I'm compiling the
% document.
% https://lists.debian.org/debian-tex-maint/2011/03/msg00101.html
% \usepackage{xfrac}
\usepackage{tikz}
\usepackage{epstopdf}
\usepackage{float}
\usepackage{stmaryrd}
\usepackage{centernot}
\usepackage{amssymb}
\usepackage{calc}
\usepackage[nomessages]{fp}
\usepackage{polynom}
\usepackage{ctable}
\usepackage[sharp]{easylist}
\usepackage{siunitx}
\usepackage{pdfpages}
\usepackage{enumerate}
\usepackage{algorithm}
\usepackage{algpseudocode}
\usepackage{pifont}
\usepackage{hyperref}
\usepackage{placeins}
\usepackage[makeroom]{cancel}
%\usepackage{sagetex}

\geometry{a4paper,left=3cm, right=3cm, top=3cm, bottom=3cm}
\pagestyle{fancy}
\fancyhead[C]{Codierungstheorie}
\fancyhead[R]{\today}
\fancyfoot[L]{Simon Wegendt}
\fancyfoot[C]{Seite \thepage /\pageref{LastPage}}
\fancyfoot[R]{David Binder}
\renewcommand{\headrulewidth}{0.4pt}
\renewcommand{\footrulewidth}{0.4pt}
\newcommand{\cmark}{\text{\ding{51}}}%
\newcommand{\xmark}{\text{\ding{55}}}%
\parindent0pt
%\newcommand{\N}{\mbox{$I\!\!N$}}
%\newcommand{\Z}{\mbox{$Z\!\!\!Z$}}
%\newcommand{\Q}{\mbox{$I\:\!\!\!\!\!Q$}}
%\newcommand{\R}{\mbox{$I\!\!R$}}
\newcommand{\N}{\mathbb{N}}
\newcommand{\Z}{\mathbb{Z}}
\newcommand{\Q}{\mathbb{Q}}
\newcommand{\R}{\mathbb{R}}
\newcommand{\C}{\mathbb{C}}
\newcommand{\D}{\mathbb{D}}
\renewcommand{\L}{\mathbb{L}}
\renewcommand{\P}{\mathcal{P}}
\newcommand{\set}[1]{\left\lbrace #1 \right\rbrace}
\newcommand{\tuple}[1]{\left( #1 \right)}
\newcommand{\entspr}{\ensuremath \widehat{=}}
\newcommand{\innervect}[1]{\begin{array}{c}#1\end{array}}
\newcommand{\vect}[1]{\tuple{\innervect{#1}}}
\makeatletter
\newcommand*{\rom}[1]{\expandafter\@slowromancap\romannumeral #1@}
\makeatother
\newcommand{\lgsto}[4]{
  \xrightarrow[
    \overset{
      \text{\scriptsize #3}
    }{
      \text{#4}
    }
  ]{
    \overset{
      \text{\scriptsize #1}
    }{
      \text{#2}
    }
  }
}
\renewcommand{\mod}{\text{ mod }}
\newcommand{\ggT}{\text{ggT}}
\newcommand{\id}{\text{id}}
\newcommand{\IV}{\overset{\text{IV}}{=}}
\newcommand{\ZZ}{\ensuremath{\mathrm{Z\kern-.3em\raise-0.5ex\hbox{Z}}:\phantom{}}}
\newcommand{\grad}{\text{grad}}
\newcommand{\phantomn}{\phantom{}}
\newcommand{\overtext}[2]{\overset{\text{\scriptsize #1}}{#2}}
\newcommand{\Abb}{\text{Abb}}
\newcommand{\seilpmi}{\hspace{3pt}\Longleftarrow\hspace{3pt}}
\newcommand{\Kern}{\text{Kern}}
\newcommand{\Bild}{\text{Bild}}
\newcommand{\Spann}[1]{\left\langle#1\right\rangle}
\newcommand{\Rang}{\text{Rang}}
\renewcommand{\phi}{\varphi}
\newcommand{\ceil}[1]{\left\lceil#1\right\rceil}
\newcommand{\floor}[1]{\left\lfloor#1\right\rfloor}
\newcommand{\TM}[1]{\left\langle#1\right\rangle}
\newcommand{\ot}{\reflectbox{$\to$}}
\newcommand{\underscore}{\underline{\hspace{5pt}}}
\newcommand{\VR}{\text{VR}}

\DeclareMathOperator*{\argmax}{argmax\,}
\DeclareMathOperator*{\argmin}{argmin\,}

\def\splitstring#1#2{%
    \noexpandarg
    \IfSubStr{#1}{#2}{
    \StrBefore{#1}{#2}
    }{#1}
}
\newcommand{\firstline}[1]{\splitstring{#1}{\\}}

\newcommand{\autobox}[1]{\parbox{\widthof{\firstline{#1}}}{#1}}
\newenvironment{myList}{
  \begin{easylist}[enumerate]\ListProperties(Numbers1=l, Mark1={)}, Style1*=\bfseries\large, Numbers2=a, Numbers3=R, Numbers4=r, Hide3=2, Mark2=., Style2*={}, Progressive=1em)
}{\end{easylist}}

\newcommand*{\independent}{\ensuremath{\bot\hspace{-0.5em}\bot}}
\newcommand*{\given}{\,|\,}


\fancyhead[L]{Übungsblatt 5}
\setcounter{MaxMatrixCols}{20}
\begin{document}
\section*{Aufgabe 22}
(David)
\begin{myList}
#
Sei $m \in \N$. Geben Sie ein Verfahren an (nicht BruteForce!), wie man zu jedem $z \in \Z^{2^m}_2$ ein Bool'sches Polynom $f$ in $m$ Variablen $x_1,\ldots,x_m$ bestimmen kann mit $\underline{f} = z$. Begründen Sie, warum Ihr Verfahren funktioniert.\medskip

TODO

#
Wenden Sie ihr Verfahren auf $z = (1,0,1,0,1,0,0,1,0,1,0,1,0,1,1,0) \in \Z^{16}_{2}$ an.\medskip

TODO

#
In welchen $RM(r,4)$, $0 < r < 4$ ist $z$ aus b enthalten?\medskip

TODO
\end{myList}

\section*{Aufgabe 23}
Bei der Übertragung eines Wortes aus $RM(1,3)$ sei maximal ein Fehler aufgetreten und $(1,0,1,1,1,0,0,1)$ wurde empfangen. Bestimmen Sie das gesendete Wort mit Majority-Logic-Decodierung.\medskip

Für jeden 1-dim. UR $M^i$ von $\Z_2^3$ bestimmen wir 2 2-dim. UR $M^i_1, M^i_2$, die sich in $M^i$ schneiden. Dabei geben wir jeweils nur eine Erzeugerbasis an:
\begin{align*}
z&=(1,0,1,1,1,0,0,1)\\
M^1&=<(1, 0, 0)>&M^2&=<(0, 1, 0)>\\
M^3&=<(1, 1, 0)>&M^4&=<(0, 0, 1)>\\
M^5&=<(1, 0, 1)>&M^6&=<(0, 1, 1)>\\
M^7&=<(1, 1, 1)>\\\\
M^1_1&=<(1, 0, 0), (0, 1, 0)>&M^1_2&=<(1, 0, 0), (0, 0, 1)>\\
M^2_1&=<(1, 0, 0), (0, 1, 0)>&M^2_2&=<(0, 1, 0), (0, 0, 1)>\\
M^3_1&=<(1, 0, 0), (0, 1, 0)>&M^3_2&=<(1, 1, 0), (0, 0, 1)>\\
M^4_1&=<(1, 0, 0), (0, 0, 1)>&M^4_2&=<(0, 1, 0), (0, 0, 1)>\\
M^5_1&=<(1, 0, 0), (0, 0, 1)>&M^5_2&=<(0, 1, 0), (1, 0, 1)>\\
M^6_1&=<(1, 0, 0), (0, 1, 1)>&M^6_2&=<(0, 1, 0), (0, 0, 1)>\\
M^7_1&=<(1, 0, 0), (0, 1, 1)>&M^7_2&=<(0, 1, 0), (1, 0, 1)>\\\\
\chi_{M^1_1}&=(1, 1, 1, 1, 0, 0, 0, 0)&\chi_{M^1_2}&=(1, 1, 0, 0, 1, 1, 0, 0)\\
\chi_{M^2_1}&=(1, 1, 1, 1, 0, 0, 0, 0)&\chi_{M^2_2}&=(1, 0, 1, 0, 1, 0, 1, 0)\\
\chi_{M^3_1}&=(1, 1, 1, 1, 0, 0, 0, 0)&\chi_{M^3_2}&=(1, 0, 0, 1, 1, 0, 0, 1)\\
\chi_{M^4_1}&=(1, 1, 0, 0, 1, 1, 0, 0)&\chi_{M^4_2}&=(1, 0, 1, 0, 1, 0, 1, 0)\\
\chi_{M^5_1}&=(1, 1, 0, 0, 1, 1, 0, 0)&\chi_{M^5_2}&=(1, 0, 1, 0, 0, 1, 0, 1)\\
\chi_{M^6_1}&=(1, 1, 0, 0, 0, 0, 1, 1)&\chi_{M^6_2}&=(1, 0, 1, 0, 1, 0, 1, 0)\\
\chi_{M^7_1}&=(1, 1, 0, 0, 0, 0, 1, 1)&\chi_{M^7_2}&=(1, 0, 1, 0, 0, 1, 0, 1)\\\\
\end{align*}
\begin{align*}
k^i_j&=\chi^i_j\cdot z\\\\
k^1_1&=1&k^1_2=0\\
k^2_1&=1&k^2_2=1\\
k^3_1&=1&k^3_2=0\\
k^4_1&=0&k^4_2=1\\
k^5_1&=0&k^5_2=1\\
k^6_1&=0&k^6_2=1\\
k^7_1&=0&k^7_2=1\\
\end{align*}
Jetzt können wir bestimmen, ob an den Stellen $(0, i)$ eine (un)gerade Anzahl Fehler aufgetreten ist, für $i=1\ldots 7$. Nur an Stellen $(0, 2)$ ist eine ungerade Anzahl Fehler aufgetreten, an den restlichen Stellenpaaren eine gerade Anzahl.\\
Im nächsten Schritt ist die Dimension der UR 0, wir können nun also bestimmen, ob an Stelle 0 eine (un)gerade Anzahl Fehler aufgetreten ist: Sie ist gerade, da nicht die Mehrzahl der Fehler an $(0, i)$ ungerade ist. Also ist an allen Stellen außer an Stelle 2 kein Fehler aufgetreten. Wir können bis zu einen Fehler korrigieren, also decodieren wir $z$ zu $$y = (1, 0, 0, 1, 1, 0, 0, 1)$$
Quellcode für diese Aufgabe (ja, wir haben MLD implementiert): \url{https://github.com/plneappl/HaskellVectorsETC}

\section*{Aufgabe 24}
Sei $m \in \N$. Zeigen Sie:
\begin{equation*}
	RM(m-1,m) = \lbrace z \in \Z^{2^m}_2 : wt(z) \text{ ist gerade } \rbrace
\end{equation*}

Zeige zunächst per Induktionsbeweis über $m$ dass $RM(m-1,m) \subseteq \lbrace z \in \Z^{2^m}_2 : wt(z) \text{ ist gerade } \rbrace$:
\begin{itemize}
	\item[IA:] Die einzigen Polynome von Grad $\leq 0$ in $RM(0,1)$ sind 0 und 1, und für 0 und 1 gilt die Aussage offensichtlich.
	\item[IH:] Sei $c = \underline{f} \in RM(n-1,n)$, dann $wt(c)$ ist gerade.
	\item[IS:] Sei $c = \underline{f} \in RM(n,n+1)$, zeige $wt(c)$ ist gerade.\\
	$f$ kann auf folgende Weise dargestellt werden:\\
	$f = g(x_0,\ldots,x_n) + x_{n+1}\cdot h(x_0,\ldots x_n)$\\
	Dann setzt sich $\underline{f}$ wie in der folgenden Darstellung zusammen:\\
	\begin{tabular}{|c|ccc|ccc|}
	\hline 
	g & $\cdots$ & $\underline{g}$ & $\cdots$ & $\cdots$ & $\underline{g}$ & $\cdots$ \\ 
	\hline 
	 &  & + &  &  & + &  \\ 
	\hline 
	$x_{n+1}$ & 0 & $\cdots$ & 0 & 1 & $\cdots$ & 1 \\ 
	\hline 
	h & $\cdots$ & $\underline{h}$ & $\cdots$ & $\cdots$ & $\underline{h}$ & $\cdots$ \\ 
	\hline
	\end{tabular}
	
	$\underline{f}$ hat also die Form $(\underline{g}, \underline{g+h})$(falls $x_{n+1}$ Most significant Bit in der Anordnung der Vektoren aus $\Z_2^{n+1}$ ist, ansonsten entsprechend permutiert).\medskip
	
	$h(x_0,\ldots,x_n)$ muss einen Grad $\leq n-1$ haben, und hat damit nach IH ein gerades Gewicht.\medskip
	
	Unterscheide jetzt die beiden Fälle dass $wt(\underline{g})$ gerade oder ungerade ist.
	Ist $wt(\underline{g})$ gerade, so ist auch $wt(\underline{g+h})$ gerade, und damit auch $wt(\underline{f})$ gerade.
	Ist $wt(\underline{g})$ ungerade, so ist auch $wt(\underline{g+h})$ ungerade, und damit aber $wt(\underline{f})$ gerade.
	(Die Implikationen ergeben sich aus der Tatsache dass $wt(\underline{g + h}) = wt(\underline{g}) + wt(\underline{h}) - 2 \cdot i$ wobei $i$ die Anzahl der Stellen ist an denen beide Vektoren $1$ sind.
\end{itemize}

Zeige nun $\lbrace z \in \Z^{2^m}_2 : wt(z) \text{ ist gerade } \rbrace \subseteq RM(m-1,m)$:

$M = \lbrace z \in \Z^{2^m}_2 : wt(z) \text{ ist gerade }\rbrace$ ist ein $m-1$ dimensionaler Unterraum des $\Z_2^{2^m}$.
(Abschlusseigenschaften sind offensichtlich, Dimension ergibt sich aus der Kardinalität).
Damit gilt nach Satz 4.15: $\chi_M \in RM(r,m)$.


\section*{Aufgabe 25}
\begin{myList}
#
Seien $\C_1$ und $\C_2$ zwei lineare Codes in $K^n$, $K$ endlicher Körper, mit $dim(\C_i) = k_i$, $i = 1,2$.\\
Sei $\C_1 \propto \C_2 \subseteq K^{2n}$ definiert durch:
\begin{equation*}
	\C_1 \propto \C_2 = \lbrace x = (c_1, c_1 + c_2) : c_1 \in \C_1, c_2 \in \C_2 \rbrace
\end{equation*}
Zeigen Sie, dass $\C_1 \propto \C_2$ ein linearer Code der Dimension $k_1 + k_2$ ist.\medskip
## $\C_1 \propto \C_2$ ist linear:\\
### Da $\C_1$ und $\C_2$ linear sind, enthalten sie mindestens jeweils einen Vektor $c_1$ und $c_2$. Also enthält auch $\C_1 \propto \C_2$ mindestens einen Vektor: $(c_1, c_1+c_2)$.
### Seien $x, y\in\C_1 \propto \C_2$ beliebig, d.h. $x=(x_1, x_1+x_2), y=(y_1, y_1+y_2)$ mit $x_1, y_1\in\C_1, x_2, y_2\in\C_2$. Sei $k\in K$ beliebig. Dann gilt:
#### \begin{align*}
x + y &= (x_1+y_1, x_1+x_2+y_1+y_2) = (x_1+y_1, x_1+y_1+x_2+y_2)\\
x + y &\in \C_1 \propto \C_2 \iff x_1+y_1\in\C_1, x_2+y_2\in\C_2
\intertext{Da $\C_1$ und $\C_2$ linear sind, gilt:}
x+y &\in \C_1 \propto \C_2
\end{align*}
#### \begin{align*}
k\cdot x &= (k x_1, k (x_1+x_2)) = (k x_1, k x_1+ k x_2)\\
k\cdot x \in \C_1 \propto \C_2 \iff k x_1\in\C_1, k x_2\in\C_2
\intertext{Da $\C_1$ und $\C_2$ linear sind, gilt:}
k\cdot x \in \C_1 \propto \C_2
\end{align*}
\ListProperties(Hide3=3)
### Da gezeigt wurde, dass $\C_1 \propto \C_2$ nicht-leer und abgeschlossen bezüglich Vektoraddition und Skalarmultiplikation ist, ist $\C_1 \propto \C_2$ ein linearer Code. \ListProperties(Hide3=2)
## $\dim(\C_1 \propto \C_2) = k_1+k_2$:\medskip\\
Seien
\begin{align*}
\C'_1&=\set{(c, c) | c\in\C_1}~~\subseteq K^{2n}\\
\C'_2&=\lbrace(\underbrace{0,\ldots, 0}_{n\text{ viele}}, c) | c\in\C_2\rbrace~~\subseteq K^{2n}
\intertext{Da $|\C'_1|=|\C_1|, |\C'_2|=|\C_2|$, bleibt die Dimension der gestrichenen Mengen gleich:}
\dim(\C'_1)&=\dim(\C_1)=k_1\\
\dim(\C'_2)&=\dim(\C_1)=k_2\\
\C'_1+\C'_2&=\C_1 \propto \C_2\\
\C'_1\cap\C'_2&=\set{(c_1, c_2)|c_1=c_2\in\C_1, c_1=0, c_2\in\C_2}\\
&=\set{(c_1, c_2)|c_1=c_2=0}=\set 0\\
&\implies\dim(\C'_1\cap\C'_2)=0
\intertext{Nach der Dimensionsformel gilt:}
\dim(\C_1 \propto \C_2)&=\dim(\C'_1+\C'_2)=\dim(\C'_1)+\dim(\C'_2)-\dim(\C'_1\cap\C'_2)=k_1+k_2-0
\end{align*}
#
Geben Sie ein Beispiel zweier zyklischer Codes $\C_1$ und $\C_2$ gleicher Länge an, sodass $C_1 \propto \C_2$ nicht zyklisch ist.\medskip

Seien $n = 2,\C_1=\set{(00)}$ und sei $\C_2=\set{(00), (11)}$ der $n$-fache Wiederholungscode. $\C_1 \propto \C_2=\set{(0000), x=(0011)}$. $\sigma(x)=(1001)\not\in\C_1 \propto \C_2$, also ist $\C_1 \propto \C_2$ nicht zyklisch. 

#
Sei $m \in \N$ und $\mathcal{W}_{2^m}=\set{\mathbf 0, \mathbf 1}$ der $2^m$-fache binäre Wiederholungscode. Zeigen Sie, dass $M_1:=RM(1,m) \propto \mathcal{W}_{2^m} = RM(1,m+1)=:M_2$.\medskip
\begin{align*}
RM(1,m) \propto \mathcal{W}_{2^m} &= \set{(c_1, c_1 + c_2) | c_1\in RM(1, m), c_2\in\mathcal{W}_{2^m}}\\
&=\set{(c, c + \mathbf 0) | c\in RM(1, m)} \dot\cup \set{(c, c + \mathbf 1) | c\in RM(1, m)}\\
&=\set{(c, c) | c\in RM(1, m)} \dot\cup \set{(c, c + \mathbf 1) | c\in RM(1, m)}
\end{align*}
## $M_1\subseteq M_2$:\medskip

Sei $\underline f\in RM(1, m)$ beliebig, $f:\Z_2^m\to\Z_2, \grad(f)\leq 1$. Zu zeigen: $(\underline f, \underline f)\in RM(1, m+1)$ und $(\underline f, \underline f + \mathbf 1)\in RM(1, m+1)$.
### $(\underline f, \underline f)\in RM(1, m+1)$:\medskip
\begin{align*}
\text{Sei }& f':\Z_2^{m+1}\to\Z_2, (x_1,\ldots, x_m, x_{m+1})\mapsto f(x_1, \ldots, x_m)\\
&\grad(f')=\grad(f)\leq 1\\
\underline{f'}&=(\underline{f'(x_1,\ldots, x_m, 0)}, \underline{f'(x_1,\ldots, x_m, 1)}) \\
&= (\underline f, \underline f)~~ \grad(f')=\grad(f)\leq 1
\intertext{Also gilt: }
(\underline f, \underline f) &= \underline{f'}\in RM(1, m+1)
\end{align*}
### $(\underline f, \underline f + \mathbf 1)\in RM(1, m+1)$:\medskip
\begin{align*}
\text{Sei }& f':\Z_2^{m+1}\to\Z_2, (x_1,\ldots, x_m, x_{m+1})\mapsto x_{m+1} + f(x_1, \ldots, x_m)\\
&\grad(f')=\max(1, \grad(f))=1\\
\underline{f'}&=(\underline{f'(x_1,\ldots, x_m, 0)}, \underline{f'(x_1,\ldots, x_m, 1)}) \\
\underline{f'}&=(\underline{0 + f(x_1, \ldots, x_m)}, \underline{1 + f(x_1, \ldots, x_m)}) \\
&= (\underline f, \underline f + \mathbf 1)
\intertext{Also gilt: }
(\underline f, \underline f + \mathbf 1) &= \underline{f'}\in RM(1, m+1)
\end{align*}
## $M_2\subseteq M_1$:\medskip

Sei $\underline f\in RM(1, m+1)$, $f:\Z_2^{m+1}\to\Z_2, \grad(f)\leq 1$, $(x_1\ldots, x_m, x_{m+1})\mapsto f(x_1\ldots, x_m, x_{m+1})$. Da der Grad von $f$ kleiner gleich 1 ist, kann jede Variable nur alleine im Polynom auftreten, insbesondere $x_{m+1}$. Doppelte Vorkommen einer Variablen $x_i$ können eliminiert werden, ohne die Abbildung zu ändern: taucht $x_i$ $2k$-mal auf, können wir $x_i$ komplett weglassen (da $x_i + x_i = 0$), taucht $x_i$ $2k+1$-mal auf, können wir $2k$ Vorkommen weglassen (da $x_i+x_i+x_i=x_i$). Wir nehmen also an, dass jede Variable in $f$ höchstens einmal vorkommt. Dann hat $f(x_1\ldots, x_m, x_{m+1})$ eine der folgenden Formen:
### $f(x_1\ldots, x_m, x_{m+1}) = g(x_1\ldots, x_m)$
### $f(x_1\ldots, x_m, x_{m+1}) = x_{m+1} + g(x_1\ldots, x_m)$
\ListProperties(Hide=10000)
## Mit $g:\Z_2^m\to\Z_2, \grad(g)\leq 1$. Dann ist also $g\in RM(1, m)$.\\
Im ersten Fall ist $\underline f$ gerade $(\underline{g}, \underline{g})\in M_1$. Im zweiten Fall ist $\underline{f} = (\underline g, \underline g + \mathbf 1) \in M_1$.
# Da beidseitige Inklusion gezeigt wurde, muss gelten: $M_1=M_2$.
\end{myList}

\section*{Aufgabe 26}
\begin{myList}
#
Bestimmen Sie alle $\alpha \in \Z_{13}$, die die multiplikative Gruppe $\Z_{13}^*$ erzeugen, d.h. $\Z_{13}^* = \lbrace \alpha^0 = 1,\alpha,\ldots,\alpha^{11} \rbrace$.\medskip

Der folgende Code testet alle $x\in[1..12]$, ob sie ein Kandidat für $\alpha$ sind, und gibt sie als Liste zurück:\\
\begin{lstlisting}
map fst $ 
  filter ((== 12) . snd) $ 
  zipWith (,) [1..12] $ 
  map length $
  map nub $ 
  map (\x -> map (\y -> (x ^ y) `mod` 13) [0..12]) [1..12]}
\end{lstlisting}
Sein Output ist: \lstinline{[2,6,7,11]}.
#
Geben Sie explizit eine Erzeugermatrix und eine Kontrollmatrix für den Reed-Solomon-Code $RS_{13}(5)$ an.\medskip

$q = 13, n = q - 1 = 12, d = 5, n - d = 7$. $\alpha=2$, $M = (\alpha^0, \alpha^1, \ldots, \alpha^{q-2})=(1, 2, 4, 8, 3, 6, 12, 11, 9, 5, 10, 7)$
## Erzeugermatrix:
$$G = \begin{pmatrix}
1 & 1 & 1 & 1 & 1 & 1 & 1 & 1 & 1 & 1 & 1 & 1\\
1 & 2 & 4 & 8 & 3 & 6 & 12 & 11 & 9 & 5 & 10 & 7\\
1 & 4 & 3 & 12 & 9 & 10 & 1 & 4 & 3 & 12 & 9 & 10\\
1 & 8 & 12 & 5 & 1 & 8 & 12 & 5 & 1 & 8 & 12 & 5\\
1 & 3 & 9 & 1 & 3 & 9 & 1 & 3 & 9 & 1 & 3 & 9\\
1 & 6 & 10 & 8 & 9 & 2 & 12 & 7 & 3 & 5 & 4 & 11\\
1 & 12 & 1 & 12 & 1 & 12 & 1 & 12 & 1 & 12 & 1 & 12\\
1 & 11 & 4 & 5 & 3 & 7 & 12 & 2 & 9 & 8 & 10 & 6
\end{pmatrix}$$
## Kontrollmatrix:
$$H = \begin{pmatrix}
1 & 2 & 4 & 8 & 3 & 6 & 12 & 11 & 9 & 5 & 10 & 7\\
1 & 4 & 3 & 12 & 9 & 10 & 1 & 4 & 3 & 12 & 9 & 10\\
1 & 8 & 12 & 5 & 1 & 8 & 12 & 5 & 1 & 8 & 12 & 5\\
1 & 3 & 9 & 1 & 3 & 9 & 1 & 3 & 9 & 1 & 3 & 9%\\
%1 & 6 & 10 & 8 & 9 & 2 & 12 & 7 & 3 & 5 & 4 & 11
\end{pmatrix}$$
\end{myList}

\section*{Aufgabe 27}
\begin{myList}
#
Bestimmen Sie ein irreduzibles Polynom vom Grad 2 über $\Z_3$.\medskip

Die Menge der Polynome von Grad 1 über $\Z_3$ ist:
$\lbrace x,x+1,x+2,2x,2x+1,2x+2 \rbrace$

Die Menge der Polynome von Grad 2 über $\Z_3$ ist:
$\lbrace
x^2, x^2+1, x^2+2, x^2+x, x^2+x+1, x^2+x+2, x^2+2x, x^2+2x+1, x^2+2x+2,
2x^2, 2x^2+1, 2x^2+2, 2x^2+x, 2x^2+x+1, 2x^2+x+2, 2x^2+2x, 2x^2+2x+1, 2x^2+2x+2
\rbrace$\medskip

Ein Polynom von Grad 2 über $\Z_3$ ist genau dann irreduzibel wenn es keine Nullstelle in $\Z_3$ hat.
$h = x^2 + 1$ ist irreduzibel, da $0^2 + 1 = 2 \neq 0$, $1^2  + 1 = 2 \neq 0$ und $2^2 + 1 = 2 \neq 0$, $h$ also keine Nullstelle in $\Z_3$ hat.
#
Konstruieren Sie mit dem Polynom aus a) einen Körper $K$ der Ordnung 9 und geben Sie seine Multiplikationstafel an.\medskip

Betrachte $\Z_3[x]_2$, die Menge der Polynome über $\Z_3$ mit Grad $< 2$.

Diese Menge ist $\lbrace 0,1,2, x, x+1 , x+2, 2x, 2x+1, 2x+2 \rbrace$ und hat damit die gewünschte Kardinalität.

Die Addition $\oplus$ im Körper ist die normale Addition von Polynomen aus $\Z_3$.

Die Multiplikation $\odot$ im Körper ist definiert als $a \odot b =_{Def} (a \cdot b) \mod h$, wobei $\cdot$ die normale Multiplikation von Polynomen aus $\Z_3$ ist und $h$ das irreduzible Polynom aus Teil a).\medskip

Da der Körper 9 Elemente hat enthält die Multiplikationstabelle 81 Einträge.
Es ist möglich die Multiplikationstabelle zu verkleinern indem man gewisse Elemente herausnimmt:

Man kann die Elemente 0,1 und 2 herausnehmen, da die Multiplikation mit diesen Elementen den Grad des Polynoms nicht erhöht, und sich die Multiplikation $\odot$ identisch zu der normalen Multiplikation $\cdot$ verhält, da gilt: Grad($a) <$ Grad($h) \Rightarrow a \mod h = a$.


Man kann desweiteren die Elemente $2x$ und $2x+2$ aus der Multiplikationstabelle entfernen, da die folgende Vereinfachung gilt (analog für $2x$):

\begin{align*}
	a \odot (2x + 2) &= a \odot (2\cdot (x+1)) \\
	&= a \cdot (2 \cdot (x+1)) \mod h \\
	&= (a \cdot (x+1)) \cdot 2 \mod h \\
	&= (a \cdot (x+1)) \mod h \cdot 2\\
	&= (a \odot (x+1)) \cdot 2\\
\end{align*}

Es muss also noch die folgende Multiplikationstabelle ausgefüllt werden (Die Tabelle ist natürlich symmetrisch zur Diagonalen).

\begin{align*}
	\begin{matrix}
	\odot & x & x+1 & x+2 & 2x+1  \\
	x & 2 & x+2 & 2x+2 & x+1   \\
	x+1 & - & 2x & 1 & x   \\
	x+2 & - & - & x & 2x   \\
	2x+1 & - & - & - & x   \\
	\end{matrix}
\end{align*}
\begin{align*}
	(x \cdot x) \mod h 			&= x^2 \mod h 			&=&  2\\ 
	(x \cdot (x+1)) \mod h 		&= x^2 + x \mod h 		&=& x+2\\
	(x \cdot (x+2)) \mod h 		&= x^2 + 2x \mod h 		&=& 2x+2\\
	(x \cdot (2x +1)) \mod h 	&= 2x^2 + x \mod h 		&=& x+1\\
	((x+1)\cdot (x+1)) \mod h 	&= x^2 + 2x + 1 \mod h 	&=& 2x\\
	((x+1)\cdot (x+2)) \mod h 	&= x^2 + 2 \mod h 		&=& 1\\
	((x+1)\cdot (2x+1)) \mod h 	&= 2x^2 + x + 2 \mod h 	&=& x\\
	((x+2)\cdot (x+2)) \mod h 	&= x^2 + x + 1 \mod h 	&=& x\\
	((x+2)\cdot (2x+1)) \mod h 	&= 2x^2 + 2x + 2 \mod h 	&=& 2x\\
	((2x+1)\cdot (2x+1)) \mod h 	&= x^2 + x + 1 \mod h 	&=& x\\
\end{align*}
#
Bestimmen Sie ein Element, das die multiplikative Gruppe $K^{\ast}$ erzeugt.\medskip

Sei $\alpha = x+2$, dann gilt:
\begin{align*}
	\alpha^0 &= 1 \\
	\alpha^1 &= x+2 \\
	\alpha^2 &= x \\
	\alpha^3 &= 2x+2 \\
	\alpha^4 &= 2 \\
	\alpha^5 &= 2x+1 \\
	\alpha^6 &= 2x \\
	\alpha^7 &= x+1 \\
	(\alpha^8 &= 1) \\
\end{align*}

Und damit erzeugt $\alpha$ offensichtlich die multiplikative Gruppe $K^\ast = \Z_3[x]_2 \setminus \lbrace 0\rbrace$.
\end{myList}
\end{document}
