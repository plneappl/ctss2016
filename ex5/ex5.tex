\documentclass{article}

\usepackage{etex}
\usepackage[utf8]{inputenc}
\usepackage{csquotes}
\usepackage{ngerman}
\usepackage{graphicx}
\usepackage{amssymb}
\usepackage{fancyhdr}
\usepackage{caption}
\usepackage{url}
\usepackage{lastpage}
\usepackage{geometry}
\usepackage{amsmath}
\usepackage{titlesec}
\usepackage{amsthm}
\usepackage{mathtools}
\usepackage{extarrows}
\usepackage{listings} \lstset{numbers=left, numberstyle=\tiny, numbersep=5pt} \lstset{language=Haskell}
\usepackage{stmaryrd}
%Felix: There is a conflict between mathtools and xfraf whem I'm compiling the
% document.
% https://lists.debian.org/debian-tex-maint/2011/03/msg00101.html
% \usepackage{xfrac}
\usepackage{tikz}
\usepackage{epstopdf}
\usepackage{float}
\usepackage{stmaryrd}
\usepackage{centernot}
\usepackage{amssymb}
\usepackage{calc}
\usepackage[nomessages]{fp}
\usepackage{polynom}
\usepackage{ctable}
\usepackage[sharp]{easylist}
\usepackage{siunitx}
\usepackage{pdfpages}
\usepackage{enumerate}
\usepackage{algorithm}
\usepackage{algpseudocode}
\usepackage{pifont}
\usepackage{hyperref}
\usepackage{placeins}
\usepackage[makeroom]{cancel}
%\usepackage{sagetex}

\geometry{a4paper,left=3cm, right=3cm, top=3cm, bottom=3cm}
\pagestyle{fancy}
\fancyhead[C]{Codierungstheorie}
\fancyhead[R]{\today}
\fancyfoot[L]{Simon Wegendt}
\fancyfoot[C]{Seite \thepage /\pageref{LastPage}}
\fancyfoot[R]{David Binder}
\renewcommand{\headrulewidth}{0.4pt}
\renewcommand{\footrulewidth}{0.4pt}
\newcommand{\cmark}{\text{\ding{51}}}%
\newcommand{\xmark}{\text{\ding{55}}}%
\parindent0pt
%\newcommand{\N}{\mbox{$I\!\!N$}}
%\newcommand{\Z}{\mbox{$Z\!\!\!Z$}}
%\newcommand{\Q}{\mbox{$I\:\!\!\!\!\!Q$}}
%\newcommand{\R}{\mbox{$I\!\!R$}}
\newcommand{\N}{\mathbb{N}}
\newcommand{\Z}{\mathbb{Z}}
\newcommand{\Q}{\mathbb{Q}}
\newcommand{\R}{\mathbb{R}}
\newcommand{\C}{\mathcal{C}}
\newcommand{\D}{\mathbb{D}}
\renewcommand{\L}{\mathbb{L}}
\renewcommand{\P}{\mathcal{P}}
\newcommand{\set}[1]{\left\lbrace #1 \right\rbrace}
\newcommand{\tuple}[1]{\left( #1 \right)}
\newcommand{\entspr}{\ensuremath \widehat{=}}
\newcommand{\innervect}[1]{\begin{array}{c}#1\end{array}}
\newcommand{\vect}[1]{\tuple{\innervect{#1}}}
\makeatletter
\newcommand*{\rom}[1]{\expandafter\@slowromancap\romannumeral #1@}
\makeatother
\newcommand{\lgsto}[4]{
  \xrightarrow[
    \overset{
      \text{\scriptsize #3}
    }{
      \text{#4}
    }
  ]{
    \overset{
      \text{\scriptsize #1}
    }{
      \text{#2}
    }
  }
}
\renewcommand{\mod}{\text{ mod }}
\newcommand{\ggT}{\text{ggT}}
\newcommand{\id}{\text{id}}
\newcommand{\IV}{\overset{\text{IV}}{=}}
\newcommand{\ZZ}{\ensuremath{\mathrm{Z\kern-.3em\raise-0.5ex\hbox{Z}}:\phantom{}}}
\newcommand{\grad}{\text{grad}}
\newcommand{\phantomn}{\phantom{}}
\newcommand{\overtext}[2]{\overset{\text{\scriptsize #1}}{#2}}
\newcommand{\Abb}{\text{Abb}}
\newcommand{\seilpmi}{\hspace{3pt}\Longleftarrow\hspace{3pt}}
\newcommand{\Kern}{\text{Kern}}
\newcommand{\Bild}{\text{Bild}}
\newcommand{\Spann}[1]{\left\langle#1\right\rangle}
\newcommand{\Rang}{\text{Rang}}
\renewcommand{\phi}{\varphi}
\newcommand{\ceil}[1]{\left\lceil#1\right\rceil}
\newcommand{\floor}[1]{\left\lfloor#1\right\rfloor}
\newcommand{\TM}[1]{\left\langle#1\right\rangle}
\newcommand{\ot}{\reflectbox{$\to$}}
\newcommand{\underscore}{\underline{\hspace{5pt}}}
\newcommand{\VR}{\text{VR}}

\DeclareMathOperator*{\argmax}{argmax\,}
\DeclareMathOperator*{\argmin}{argmin\,}

\def\splitstring#1#2{%
    \noexpandarg
    \IfSubStr{#1}{#2}{
    \StrBefore{#1}{#2}
    }{#1}
}
\newcommand{\firstline}[1]{\splitstring{#1}{\\}}

\newcommand{\listSettings}{\ListProperties(Numbers1=l, Mark1={)}, Style1*=\bfseries\large, Numbers2=a, Numbers3=R, Numbers4=r, Hide3=2, Mark2=., Style2*={}, Progressive=1em)}

\newcommand{\autobox}[1]{\parbox{\widthof{\firstline{#1}}}{#1}}
\newenvironment{myList}{
  \begin{easylist}[enumerate]\listSettings
}{\end{easylist}}

\newcommand*{\independent}{\ensuremath{\bot\hspace{-0.5em}\bot}}
\newcommand*{\given}{\,|\,}


\fancyhead[L]{Übungsblatt 5}
\setcounter{MaxMatrixCols}{20}
\begin{document}
\section*{Aufgabe 22}
(David)
\begin{myList}
#
Sei $m \in \N$. Geben Sie ein Verfahren an (nicht BruteForce!), wie man zu jedem $z \in \Z^{2^m}_2$ ein Bool'sches Polynom $f$ in $m$ Variablen $x_1,\ldots,x_m$ bestimmen kann mit $\underline{f} = z$. Begründen Sie, warum Ihr Verfahren funktioniert.\medskip

TODO

#
Wenden Sie ihr Verfahren auf $z = (1,0,1,0,1,0,0,1,0,1,0,1,0,1,1,0) \in \Z^{16}_{2}$ an.\medskip

TODO

#
In welchen $RM(r,4)$, $0 < r < 4$ ist $z$ aus b enthalten?\medskip

TODO
\end{myList}

\section*{Aufgabe 23}
(Simon)
Bei der Übertragung eines Wortes aus $RM(1,3)$ sei maximal ein Fehler aufgetreten und $(1,0,1,1,1,0,0,1)$ wurde empfangen. Bestimmen Sie das gesendete Wort mit Majority-Logic-Decodierung.\medskip

TODO

\section*{Aufgabe 24}
(David)
Sei $m \in \N$. Zeigen Sie:
\begin{equation*}
	RM(m-1,m) = \lbrace z \in \Z^{2^m}_2 : wt(z) \text{ ist gerade } \rbrace
\end{equation*}

\begin{itemize}
	\item[IA:] yada yada
	\item[IH:] Sei $c = \underline{f} \in RM(n-1,n)$, dann $wt(c)$ ist gerade.
	\item[IS:] Sei $c = \underline{f} \in RM(n,n+1)$, zeige $wt(c)$ ist gerade.\\
	$f$ kann auf folgende Weise dargestellt werden:\\
	$f = g(x_0,\ldots,x_n) + x_{n+1}\cdot h(x_0,\ldots x_n)$\\
	Dann setzt sich $\underline{f}$ wie in der folgenden Darstellung zusammen:\\
	\begin{tabular}{|c|ccc|ccc|}
	\hline 
	g & $\cdots$ & $\underline{g}$ & $\cdots$ & $\cdots$ & $\underline{g}$ & $\cdots$ \\ 
	\hline 
	 &  & + &  &  & + &  \\ 
	\hline 
	$x_{n+1}$ & 1 & $\cdots$ & 1 & 0 & $\cdots$ & 0 \\ 
	\hline 
	h & $\cdots$ & $\underline{h}$ & $\cdots$ & $\cdots$ & $\underline{h}$ & $\cdots$ \\ 
	\hline
	\end{tabular}
	
	Wir zeigen jetzt durch Fallunterscheidungen dass das entstehende $\underline{f}$ immer ein gerades Gewicht hat.
	\begin{itemize}
		\item Fall 1: $h = 0 \Rightarrow f = g'(x_0, \ldots x_n,x_{n+1})$ ($g'$ ist die offensichtliche Erweiterung des Ursprungsbereiches von $g$).\\
		Es ist klar dass $g'(\vec{x},0) = g'(\vec{x},1)$, und damit ist $wt(\underline{g'}) = wt(\underline{f})$ gerade.
		\item Fall $2a$: $h \neq 0, g \neq 0$
			\begin{itemize}
			\item Fall $2a'$: $g + h = 0$\\
			Damit $g + h = 0$ muss gelten $g = h$(Definition XOR), und da $wt(\underline{h})$ gerade ist nach IV ist damit auch $wt(\underline{g})$ gerade, und damit auch $wt(\underline{f})$.
			\item Fall $2a''$:$g + h \neq 0$\\
			Entweder stimmt $g$ mit $h$ an gerade oder ungerade vielen Stellen nicht überein.
			Wenn $g$ mit $h$ an ungerade vielen Stellen nicht übereinstimmt, dann ist sowohl $wt(g+h)$ als auch $wt(g)$ ungerade, und damit $wt(f)$ gerade.
			
			Wenn $g$ mit $h$ an gerade vielen Stellen nicht übereinstimmt, dann ist sowohl $wt(g+h)$ als auch $wt(g)$ gerade, und damit $wt(f)$ gerade.
			\end{itemize}
		\item Fall 2b: $h \neq 0, g = 0\Rightarrow f = x_{n+1}\cdot h(x_0,\ldots,x_n)$
		Grad$(h) \leq n-1$, also $wt(\underline{h})$ ist gerade nach IH.
		Also ist $wt(\underline{f})$ gerade.
	\end{itemize}
\end{itemize}

\section*{Aufgabe 25}
(Simon)
\begin{myList}
#
Seien $\C_1$ und $\C_2$ zwei lineare Codes in $K^n$, $K$ endlicher Körper, mit $dim(\C_i) = k_i$, $i = 1,2$.\\
Sei $\C_1 \propto \C_2 \subseteq K^{2n}$ definiert durch:
\begin{equation*}
	\C_1 \propto \C_2 = \lbrace x = (c_1, c_1 + c_2) : c_1 \in \C_1, c_2 \in \C_2 \rbrace
\end{equation*}
Zeigen Sie, dass $\C_1 \propto \C_2$ ein linearer Code der Dimension $k_1 + k_2$ ist.\medskip

TODO
#
Geben Sie ein Beispiel zweier zyklischer Codes $\C_1$ und $\C_2$ gleicher Länge an, sodass $C_1 \propto \C_2$ nicht zyklisch ist.\medskip

TODO
#
Sei $m \in \N$ und $\mathcal{W}_{2^m}$ der $2^m$-fache binäre Wiederholungscode. Zeigen Sie, dass $RM(1,m) \propto \mathcal{W}_{2^m} = RM(1,m+1)$.\medskip

TODO
\end{myList}

\section*{Aufgabe 26}
(Simon)
\begin{myList}
#
Bestimmen Sie alle $\alpha \in \Z_{13}$, die die multiplikative Gruppe $\Z_{13}^{\ast}$ erzeugen, d.h. $\Z_{13}^{\ast} = \lbrace \alpha^0 = 1,\alpha,\ldots,\alpha^{11} \rbrace$.
\medskip

TODO
#
Geben Sie explizit eine Erzeugermatrix und eine Kontrollmatrix für den Reed-Solomon-Code $RS_{13}(5)$ an.\medskip

TODO
\end{myList}

\section*{Aufgabe 27}
\begin{myList}
#
Bestimmen Sie ein irreduzibles Polynom vom Grad 2 über $\Z_3$.\medskip

Die Menge der Polynome von Grad 1 über $\Z_3$ ist:
$\lbrace x,x+1,x+2,2x,2x+1,2x+2 \rbrace$

Die Menge der Polynome von Grad 2 über $\Z_3$ ist:
$\lbrace
x^2, x^2+1, x^2+2, x^2+x, x^2+x+1, x^2+x+2, x^2+2x, x^2+2x+1, x^2+2x+2,
2x^2, 2x^2+1, 2x^2+2, 2x^2+x, 2x^2+x+1, 2x^2+x+2, 2x^2+2x, 2x^2+2x+1, 2x^2+2x+2
\rbrace$\medskip

Ein Polynom von Grad 2 über $\Z_3$ ist genau dann irreduzibel wenn es keine Nullstelle in $\Z_3$ hat.
$h = x^2 + 1$ ist irreduzibel, da $0^2 + 1 = 2 \neq 0$, $1^2  + 1 = 2 \neq 0$ und $2^2 + 1 = 2 \neq 0$, $h$ also keine Nullstelle in $\Z_3$ hat.
#
Konstruieren Sie mit dem Polynom aus a) einen Körper $K$ der Ordnung 9 und geben Sie seine Multiplikationstafel an.\medskip

Betrachte $\Z_3[x]_2$, die Menge der Polynome über $\Z_3$ mit Grad $< 2$.

Diese Menge ist $\lbrace 0,1,2, x, x+1 , x+2, 2x, 2x+1, 2x+2 \rbrace$ und hat damit die gewünschte Kardinalität.

Die Addition $\oplus$ im Körper ist die normale Addition von Polynomen aus $\Z_3$.

Die Multiplikation $\odot$ im Körper ist definiert als $a \odot b =_{Def} (a \cdot b) \mod h$, wobei $\cdot$ die normale Multiplikation von Polynomen aus $\Z_3$ ist und $h$ das irreduzible Polynom aus Teil a).\medskip

Da der Körper 9 Elemente hat enthält die Multiplikationstabelle 81 Einträge.
Es ist möglich die Multiplikationstabelle zu verkleinern indem man gewisse Elemente herausnimmt:

Man kann die Elemente 0,1 und 2 herausnehmen, da die Multiplikation mit diesen Elementen den Grad des Polynoms nicht erhöht, und sich die Multiplikation $\odot$ identisch zu der normalen Multiplikation $\cdot$ verhält, da gilt: Grad($a) <$ Grad($h) \Rightarrow a \mod h = a$.


Man kann desweiteren die Elemente $2x$ und $2x+2$ aus der Multiplikationstabelle entfernen, da die folgende Vereinfachung gilt (analog für $2x$):

\begin{align*}
	a \odot (2x + 2) &= a \odot (2\cdot (x+1)) \\
	&= a \cdot (2 \cdot (x+1)) \mod h \\
	&= (a \cdot (x+1)) \cdot 2 \mod h \\
	&= (a \cdot (x+1)) \mod h \cdot 2\\
	&= (a \odot (x+1)) \cdot 2\\
\end{align*}

Es muss also noch die folgende Multiplikationstabelle ausgefüllt werden (Die Tabelle ist natürlich symmetrisch zur Diagonalen).

\begin{align*}
	\begin{matrix}
	\odot & x & x+1 & x+2 & 2x+1  \\
	x & 2 & x+2 & 2x+2 & x+1   \\
	x+1 & - & 2x & 1 & x   \\
	x+2 & - & - & x & 2x   \\
	2x+1 & - & - & - & x   \\
	\end{matrix}
\end{align*}
\begin{align*}
	(x \cdot x) \mod h 			&= x^2 \mod h 			&=&  2\\ 
	(x \cdot (x+1)) \mod h 		&= x^2 + x \mod h 		&=& x+2\\
	(x \cdot (x+2)) \mod h 		&= x^2 + 2x \mod h 		&=& 2x+2\\
	(x \cdot (2x +1)) \mod h 	&= 2x^2 + x \mod h 		&=& x+1\\
	((x+1)\cdot (x+1)) \mod h 	&= x^2 + 2x + 1 \mod h 	&=& 2x\\
	((x+1)\cdot (x+2)) \mod h 	&= x^2 + 2 \mod h 		&=& 1\\
	((x+1)\cdot (2x+1)) \mod h 	&= 2x^2 + x + 2 \mod h 	&=& x\\
	((x+2)\cdot (x+2)) \mod h 	&= x^2 + x + 1 \mod h 	&=& x\\
	((x+2)\cdot (2x+1)) \mod h 	&= 2x^2 + 2x + 2 \mod h 	&=& 2x\\
	((2x+1)\cdot (2x+1)) \mod h 	&= x^2 + x + 1 \mod h 	&=& x\\
\end{align*}
#
Bestimmen Sie ein Element, das die multiplikative Gruppe $K^{\ast}$ erzeugt.\medskip

Sei $\alpha = x+2$, dann gilt:
\begin{align*}
	\alpha^0 &= 1 \\
	\alpha^1 &= x+2 \\
	\alpha^2 &= x \\
	\alpha^3 &= 2x+2 \\
	\alpha^4 &= 2 \\
	\alpha^5 &= 2x+1 \\
	\alpha^6 &= 2x \\
	\alpha^7 &= x+1 \\
	(\alpha^8 &= 1) \\
\end{align*}

Und damit erzeugt $\alpha$ offensichtlich die multiplikative Gruppe $K^\ast = \Z_3[x]_2 \setminus \lbrace 0\rbrace$.
\end{myList}
\end{document}
