\documentclass{article}

\usepackage{etex}
\usepackage[utf8]{inputenc}
\usepackage{csquotes}
\usepackage{ngerman}
\usepackage{graphicx}
\usepackage{amssymb}
\usepackage{fancyhdr}
\usepackage{caption}
\usepackage{url}
\usepackage{lastpage}
\usepackage{geometry}
\usepackage{amsmath}
\usepackage{titlesec}
\usepackage{amsthm}
\usepackage{mathtools}
\usepackage{extarrows}
\usepackage{listings} \lstset{numbers=left, numberstyle=\tiny, numbersep=5pt} \lstset{language=Haskell}
\usepackage{stmaryrd}
%Felix: There is a conflict between mathtools and xfraf whem I'm compiling the
% document.
% https://lists.debian.org/debian-tex-maint/2011/03/msg00101.html
% \usepackage{xfrac}
\usepackage{tikz}
\usepackage{epstopdf}
\usepackage{float}
\usepackage{stmaryrd}
\usepackage{centernot}
\usepackage{amssymb}
\usepackage{calc}
\usepackage[nomessages]{fp}
\usepackage{polynom}
\usepackage{ctable}
\usepackage[sharp]{easylist}
\usepackage{siunitx}
\usepackage{pdfpages}
\usepackage{enumerate}
\usepackage{algorithm}
\usepackage{algpseudocode}
\usepackage{pifont}
\usepackage{hyperref}
\usepackage{placeins}
\usepackage[makeroom]{cancel}
%\usepackage{sagetex}

\geometry{a4paper,left=3cm, right=3cm, top=3cm, bottom=3cm}
\pagestyle{fancy}
\fancyhead[C]{Codierungstheorie}
\fancyhead[R]{\today}
\fancyfoot[L]{Simon Wegendt}
\fancyfoot[C]{Seite \thepage /\pageref{LastPage}}
\fancyfoot[R]{David Binder}
\renewcommand{\headrulewidth}{0.4pt}
\renewcommand{\footrulewidth}{0.4pt}
\newcommand{\cmark}{\text{\ding{51}}}%
\newcommand{\xmark}{\text{\ding{55}}}%
\parindent0pt
%\newcommand{\N}{\mbox{$I\!\!N$}}
%\newcommand{\Z}{\mbox{$Z\!\!\!Z$}}
%\newcommand{\Q}{\mbox{$I\:\!\!\!\!\!Q$}}
%\newcommand{\R}{\mbox{$I\!\!R$}}
\newcommand{\N}{\mathbb{N}}
\newcommand{\Z}{\mathbb{Z}}
\newcommand{\Q}{\mathbb{Q}}
\newcommand{\R}{\mathbb{R}}
\newcommand{\C}{\mathcal{C}}
\newcommand{\D}{\mathbb{D}}
\renewcommand{\L}{\mathbb{L}}
\renewcommand{\P}{\mathcal{P}}
\newcommand{\set}[1]{\left\lbrace #1 \right\rbrace}
\newcommand{\tuple}[1]{\left( #1 \right)}
\newcommand{\entspr}{\ensuremath \widehat{=}}
\newcommand{\innervect}[1]{\begin{array}{c}#1\end{array}}
\newcommand{\vect}[1]{\tuple{\innervect{#1}}}
\makeatletter
\newcommand*{\rom}[1]{\expandafter\@slowromancap\romannumeral #1@}
\makeatother
\newcommand{\lgsto}[4]{
  \xrightarrow[
    \overset{
      \text{\scriptsize #3}
    }{
      \text{#4}
    }
  ]{
    \overset{
      \text{\scriptsize #1}
    }{
      \text{#2}
    }
  }
}
\renewcommand{\mod}{\text{ mod }}
\newcommand{\ggT}{\text{ggT}}
\newcommand{\id}{\text{id}}
\newcommand{\IV}{\overset{\text{IV}}{=}}
\newcommand{\ZZ}{\ensuremath{\mathrm{Z\kern-.3em\raise-0.5ex\hbox{Z}}:\phantom{}}}
\newcommand{\grad}{\text{grad}}
\newcommand{\phantomn}{\phantom{}}
\newcommand{\overtext}[2]{\overset{\text{\scriptsize #1}}{#2}}
\newcommand{\Abb}{\text{Abb}}
\newcommand{\seilpmi}{\hspace{3pt}\Longleftarrow\hspace{3pt}}
\newcommand{\Kern}{\text{Kern}}
\newcommand{\Bild}{\text{Bild}}
\newcommand{\Spann}[1]{\left\langle#1\right\rangle}
\newcommand{\Rang}{\text{Rang}}
\renewcommand{\phi}{\varphi}
\newcommand{\ceil}[1]{\left\lceil#1\right\rceil}
\newcommand{\floor}[1]{\left\lfloor#1\right\rfloor}
\newcommand{\TM}[1]{\left\langle#1\right\rangle}
\newcommand{\ot}{\reflectbox{$\to$}}
\newcommand{\underscore}{\underline{\hspace{5pt}}}
\newcommand{\VR}{\text{VR}}

\DeclareMathOperator*{\argmax}{argmax\,}
\DeclareMathOperator*{\argmin}{argmin\,}

\def\splitstring#1#2{%
    \noexpandarg
    \IfSubStr{#1}{#2}{
    \StrBefore{#1}{#2}
    }{#1}
}
\newcommand{\firstline}[1]{\splitstring{#1}{\\}}

\newcommand{\listSettings}{\ListProperties(Numbers1=l, Mark1={)}, Style1*=\bfseries\large, Numbers2=a, Numbers3=R, Numbers4=r, Hide3=2, Mark2=., Style2*={}, Progressive=1em)}

\newcommand{\autobox}[1]{\parbox{\widthof{\firstline{#1}}}{#1}}
\newenvironment{myList}{
  \begin{easylist}[enumerate]\listSettings
}{\end{easylist}}

\newcommand*{\independent}{\ensuremath{\bot\hspace{-0.5em}\bot}}
\newcommand*{\given}{\,|\,}


\fancyhead[L]{Übungsblatt 5}
\setcounter{MaxMatrixCols}{20}
\begin{document}
\section*{Aufgabe 22}
\begin{myList}
#
Sei $m \in \N$. Geben Sie ein Verfahren an (nicht BruteForce!), wie man zu jedem $z \in \Z^{2^m}_2$ ein Bool'sches Polynom $f$ in $m$ Variablen $x_1,\ldots,x_m$ bestimmen kann mit $\underline{f} = z$. Begründen Sie, warum Ihr Verfahren funktioniert.\medskip

TODO

#
Wenden Sie ihr Verfahren auf $z = (1,0,1,0,1,0,0,1,0,1,0,1,0,1,1,0) \in \Z^{16}_{2}$ an.\medskip

TODO

#
In welchen $RM(r,4)$, $0 < r < 4$ ist $z$ aus b enthalten?\medskip

TODO
\end{myList}

\section*{Aufgabe 23}
Bei der Übertragung eines Wortes aus $RM(1,3)$ sei maximal ein Fehler aufgetreten und $(1,0,1,1,1,0,0,1)$ wurde empfangen. Bestimmen Sie das gesendete Wort mit Majority-Logic-Decodierung.\medskip

TODO

\section*{Aufgabe 24}
Sei $m \in \N$. Zeigen Sie:
\begin{equation*}
	RM(m-1,m) = \lbrace z \in \Z^{2^m}_2 : wt(z) \text{ ist gerade } \rbrace
\end{equation*}

TODO

\section*{Aufgabe 25}
\begin{myList}
#
Seien $\C_1$ und $\C_2$ zwei lineare Codes in $K^n$, $K$ endlicher Körper, mit $dim(\C_i) = k_i$, $i = 1,2$.\\
Sei $\C_1 \propto \C_2 \subseteq K^{2n}$ definiert durch:
\begin{equation*}
	\C_1 \propto \C_2 = \lbrace x = (c_1, c_1 + c_2) : c_1 \in \C_1, c_2 \in \C_2 \rbrace
\end{equation*}
Zeigen Sie, dass $\C_1 \propto \C_2$ ein linearer Code der Dimension $k_1 + k_2$ ist.\medskip

TODO
#
Geben Sie ein Beispiel zweier zyklischer Codes $\C_1$ und $\C_2$ gleicher Länge an, sodass $C_1 \propto \C_2$ nicht zyklisch ist.\medskip

TODO
#
Sei $m \in \N$ und $\mathcal{W}_{2^m}$ der $2^m$-fache binäre Wiederholungscode. Zeigen Sie, dass $RM(1,m) \propto \mathcal{W}_{2^m} = RM(1,m+1)$.\medskip

TODO
\end{myList}

\section*{Aufgabe 26}
\begin{myList}
#
Bestimmen Sie alle $\alpha \in \Z_{13}$, die die multiplikative Gruppe $\Z_{13}^{\ast}$ erzeugen, d.h. $\Z_{13}^{\ast} = \lbrace \alpha^0 = 1,\alpha,\ldots,\alpha^{11} \rbrace$.
\medskip

TODO
#
Geben Sie explizit eine Erzeugermatrix und eine Kontrollmatrix für den Reed-Solomon-Code $RS_{13}(5)$ an.\medskip

TODO
\end{myList}

\section*{Aufgabe 27}
\begin{myList}
#
Bestimmen Sie ein irreduzibles Polynom vom Grad 2 über $\Z_3$.\medskip

TODO

#
Konstruieren Sie mit dem Polynom aus a) einen Körper $K$ der Ordnung 9 und geben Sie seine Multiplikationstafel an.\medskip

TODO

#
Bestimmen Sie ein Element, das die multiplikative Gruppe $K^{\ast}$ erzeugt.\medskip

TODO
\end{myList}
\end{document}
