\documentclass{article}

\usepackage{etex}
\usepackage[utf8]{inputenc}
\usepackage{csquotes}
\usepackage{ngerman}
\usepackage{graphicx}
\usepackage{amssymb}
\usepackage{fancyhdr}
\usepackage{caption}
\usepackage{url}
\usepackage{lastpage}
\usepackage{geometry}
\usepackage{amsmath}
\usepackage{titlesec}
\usepackage{amsthm}
\usepackage{mathtools}
\usepackage{extarrows}
\usepackage{listings} \lstset{numbers=left, numberstyle=\tiny, numbersep=5pt} \lstset{language=Haskell}
\usepackage{stmaryrd}
%Felix: There is a conflict between mathtools and xfraf whem I'm compiling the
% document.
% https://lists.debian.org/debian-tex-maint/2011/03/msg00101.html
% \usepackage{xfrac}
\usepackage{tikz}
\usepackage{epstopdf}
\usepackage{float}
\usepackage{stmaryrd}
\usepackage{centernot}
\usepackage{amssymb}
\usepackage{calc}
\usepackage[nomessages]{fp}
\usepackage{polynom}
\usepackage{ctable}
\usepackage[sharp]{easylist}
\usepackage{siunitx}
\usepackage{pdfpages}
\usepackage{enumerate}
\usepackage{algorithm}
\usepackage{algpseudocode}
\usepackage{pifont}
\usepackage{hyperref}
\usepackage{placeins}
\usepackage[makeroom]{cancel}
%\usepackage{sagetex}

\geometry{a4paper,left=3cm, right=3cm, top=3cm, bottom=3cm}
\pagestyle{fancy}
\fancyhead[C]{Codierungstheorie}
\fancyhead[R]{\today}
\fancyfoot[L]{Simon Wegendt}
\fancyfoot[C]{Seite \thepage /\pageref{LastPage}}
\fancyfoot[R]{David Binder}
\renewcommand{\headrulewidth}{0.4pt}
\renewcommand{\footrulewidth}{0.4pt}
\newcommand{\cmark}{\text{\ding{51}}}%
\newcommand{\xmark}{\text{\ding{55}}}%
\parindent0pt
%\newcommand{\N}{\mbox{$I\!\!N$}}
%\newcommand{\Z}{\mbox{$Z\!\!\!Z$}}
%\newcommand{\Q}{\mbox{$I\:\!\!\!\!\!Q$}}
%\newcommand{\R}{\mbox{$I\!\!R$}}
\newcommand{\N}{\mathbb{N}}
\newcommand{\Z}{\mathbb{Z}}
\newcommand{\Q}{\mathbb{Q}}
\newcommand{\R}{\mathbb{R}}
\newcommand{\C}{\mathcal{C}}
\newcommand{\D}{\mathbb{D}}
\renewcommand{\L}{\mathbb{L}}
\renewcommand{\P}{\mathcal{P}}
\newcommand{\set}[1]{\left\lbrace #1 \right\rbrace}
\newcommand{\tuple}[1]{\left( #1 \right)}
\newcommand{\entspr}{\ensuremath \widehat{=}}
\newcommand{\innervect}[1]{\begin{array}{c}#1\end{array}}
\newcommand{\vect}[1]{\tuple{\innervect{#1}}}
\makeatletter
\newcommand*{\rom}[1]{\expandafter\@slowromancap\romannumeral #1@}
\makeatother
\newcommand{\lgsto}[4]{
  \xrightarrow[
    \overset{
      \text{\scriptsize #3}
    }{
      \text{#4}
    }
  ]{
    \overset{
      \text{\scriptsize #1}
    }{
      \text{#2}
    }
  }
}
\renewcommand{\mod}{\text{ mod }}
\newcommand{\ggT}{\text{ggT}}
\newcommand{\id}{\text{id}}
\newcommand{\IV}{\overset{\text{IV}}{=}}
\newcommand{\ZZ}{\ensuremath{\mathrm{Z\kern-.3em\raise-0.5ex\hbox{Z}}:\phantom{}}}
\newcommand{\grad}{\text{grad}}
\newcommand{\phantomn}{\phantom{}}
\newcommand{\overtext}[2]{\overset{\text{\scriptsize #1}}{#2}}
\newcommand{\Abb}{\text{Abb}}
\newcommand{\seilpmi}{\hspace{3pt}\Longleftarrow\hspace{3pt}}
\newcommand{\Kern}{\text{Kern}}
\newcommand{\Bild}{\text{Bild}}
\newcommand{\Spann}[1]{\left\langle#1\right\rangle}
\newcommand{\Rang}{\text{Rang}}
\renewcommand{\phi}{\varphi}
\newcommand{\ceil}[1]{\left\lceil#1\right\rceil}
\newcommand{\floor}[1]{\left\lfloor#1\right\rfloor}
\newcommand{\TM}[1]{\left\langle#1\right\rangle}
\newcommand{\ot}{\reflectbox{$\to$}}
\newcommand{\underscore}{\underline{\hspace{5pt}}}
\newcommand{\VR}{\text{VR}}

\DeclareMathOperator*{\argmax}{argmax\,}
\DeclareMathOperator*{\argmin}{argmin\,}

\def\splitstring#1#2{%
    \noexpandarg
    \IfSubStr{#1}{#2}{
    \StrBefore{#1}{#2}
    }{#1}
}
\newcommand{\firstline}[1]{\splitstring{#1}{\\}}

\newcommand{\listSettings}{\ListProperties(Numbers1=l, Mark1={)}, Style1*=\bfseries\large, Numbers2=a, Numbers3=R, Numbers4=r, Hide3=2, Mark2=., Style2*={}, Progressive=1em)}

\newcommand{\autobox}[1]{\parbox{\widthof{\firstline{#1}}}{#1}}
\newenvironment{myList}{
  \begin{easylist}[enumerate]\listSettings
}{\end{easylist}}

\newcommand*{\independent}{\ensuremath{\bot\hspace{-0.5em}\bot}}
\newcommand*{\given}{\,|\,}


\fancyhead[L]{Übungsblatt 2}
\begin{document}
\section*{Aufgabe 6}
Die Quelle $Q$ gibt die Zeichen 0 und 1 mit den Wahrscheinlichkeiten $p_Q(0) = \frac{1}{4}$ und $p_Q(1) = \frac{3}{4}$ aus.
Die Zeichen werden über einen binär symmetrischen Kanal mit $p = \frac{1}{16}$ übertragen.
\medskip
\begin{myList}

# Berechnung der Entropie der Quelle:
\begin{align*} 
H(Q) 
  &= \sum_{a\in A} p_Q(a)\cdot\log_2\tuple{\frac 1 {p_Q(a)}} \\
  &= \frac 1 4\log(4) + \frac 3 4\log\tuple{\frac 4 3} \\
  &\approx 0.5+0.311=0.811
\end{align*}

# Berechnung der Outputentropie:
\begin{align*}
p_E(0) &= \frac 1 4\frac{15}{16} + \frac 3 4\frac 1 {16} = \frac 9{32}\\
p_E(1) &= 1 - p_E(0) = \frac{23}{32}\\
H(E) &= \frac{9}{32}\log\tuple{\frac{32}{9}} + \frac{23}{32}\log\tuple{\frac{32}{23}}\\
&\approx 0.857
\end{align*}

# Berechnung der Streuentropie:
\begin{align*}
H(E|Q)&=H_p&&\text{(Skript 2.16)}\\
&= \frac 1 {16}\log(16) + \frac{15}{16}\log\tuple{\frac{16}{15}}\\
&\approx 0.337
\end{align*}

# Berechnung der mittleren Transinformation:
$$I(Q, E) = H(E) - H(E|Q) \approx 0.857 - 0.337 = 0.520 

\end{myList}

\section*{Aufgabe 7}
\begin{myList}
#
$A = \lbrace 0,1 \rbrace$, $B = \lbrace 0,1,\ast \rbrace$
\begin{align*}
	p(0|0) &= \frac{1}{2} & p(1|0) &= \frac{1}{4} & p(\ast|0) &= \frac{1}{4} \\
	p(1|1) &= \frac{1}{2} & p(0|1) &= \frac{1}{4} & p(\ast|1) &= \frac{1}{4} 
\end{align*}

Berechne $I(Q,E)$ in Abhängigkeit von $p_Q$:\\
TODO

Maximiere die Funktion:\\
TODO
#
$A = B = \lbrace 0,1,2 \rbrace$
\begin{align*}
	p(0|0) &= p & p(1|0) &= (1-p) & p(2|0) &= 0 \\
	p(1|1) &= p & p(0|1) &= (1-p) & p(2|1) &= 0 \\
	p(2|2) &= 1 & p(0|2) &= 0 & p(1|2) &= 0
\end{align*}

Berechne $I(Q,E)$ in Abhängigkeit von $p_Q$:\\
TODO

Maximiere die Funktion:\\
TODO
\end{myList}

\section*{Aufgabe 8}
\begin{myList}
#
Zeige $H(E|Q) = H(Q,E) - H(Q)$ und $H(E|Q) \leq H(E)$.

TODO

#
Zeige $H(Q|E) = H(Q,E) - H(E)$

TODO

#
Zeige $I(Q,E) = H(Q) - H(Q|E) \leq H(Q)$

TODO

\end{myList}

\section*{Aufgabe 9}

Sei die Codierungsfunktion $c: \mathbb{Z}^3_2 \rightarrow \mathbb{Z}^5_2 , (x,y,z) \mapsto (x,y,z,x+z,y+z)$.\\
Sei $\mathcal{C} = \lbrace v \in \mathbb{Z}^5_2 | \exists w \in \mathbb{Z}^3_2: v = c(w) \rbrace$.
\begin{myList}
#
Zeige dass $\mathcal{C}$ ein linearer Code ist.

TODO

#
Geben Sie eine Erzeugermatrix von $\mathcal{C}$ an.

TODO

#
Bestimmen Sie die Länge $n$, die Dimension $k$ und den Minimalabstand $d$ von $\mathcal{C}$.

TODO

#
Wieviele Fehler kann $\mathcal{C}$ erkennen /  korrigieren?

TODO
\end{myList}

\section*{Aufgabe 10}
\begin{myList}
#
Schreiben Sie ein Programm, das zufällig einen 4-dimensionalen Untervektorraum des $\mathbb{Z}^10_2$ erzeugt und den Minimalabstand von $U$ berechnet.
Welchen durchschnittlichen Minimalabstand erhalten Sie nach 100 Programmdurchläufen?\\
Im Schnitt ist der Minimalabstand 2.9987 (über 10000 Iterationen).

TODO

#
Konstruieren Sie einen binären $[10,4]$-Code mit möglichst großen Minimalabstand.\\
Bester von uns gefundener Code:\\
\begin{align*}
\C = \tuple{\begin{array}{ccc}
0101&0110&01\\
0011&1001&01\\
1100&1000&00\\
1011&0010&10
\end{array}},~~d(\C)=6
\end{align*}

TODO
\end{myList}\end{document}





